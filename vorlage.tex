%%%%%%%%%%%%%%%%%%% vorlage.tex %%%%%%%%%%%%%%%%%%%%%%%%%%%%%
%
% LaTeX-Vorlage zur Erstellung von Projekt-Dokumentationen
% im Fachbereich Informatik der Hochschule Trier
%
% Basis: Vorlage svmono des Springer Verlags
%
%%%%%%%%%%%%%%%%%%%%%%%%%%%%%%%%%%%%%%%%%%%%%%%%%%%%%%%%%%%%%

\documentclass[envcountsame,envcountchap, deutsch]{i-studis}

\usepackage{makeidx}         	% Index
\usepackage{multicol}        	% Zweispaltiger Index
%\usepackage[bottom]{footmisc}	% Erzeugung von Fußnoten

%%-----------------------------------------------------
%\newif\ifpdf
%\ifx\pdfoutput\undefined
%\pdffalse
%\else
%\pdfoutput=1
%\pdftrue
%\fi
%%--------------------------------------------------------
%\ifpdf
\usepackage[pdftex]{graphicx}
\usepackage{epstopdf}
\usepackage[pdftex,plainpages=false]{hyperref}
%\else
%\usepackage{graphicx}
%\usepackage[plainpages=false]{hyperref}
%\fi

%%-----------------------------------------------------
\usepackage{color}				% Farbverwaltung
%\usepackage{ngerman} 			% Neue deutsche Rechtsschreibung
\usepackage[english, ngerman]{babel}
%\usepackage[latin1]{inputenc} 	% Ermöglicht Umlaute-Darstellung
\usepackage[utf8]{inputenc}  	% Ermöglicht Umlaute-Darstellung unter Linux (je nach verwendetem Format)
\usepackage[T1]{fontenc}

%-----------------------------------------------------
\definecolor{red}{rgb}{0.6,0,0}

\usepackage{listings} 			% Code-Darstellung
\lstset
{
	basicstyle=\scriptsize, 	% print whole listing small
	keywordstyle=\color{blue}\bfseries,
								% underlined bold black keywords
	identifierstyle=, 			% nothing happens
	commentstyle=\color{green}, 	% white comments	
	showstringspaces=false, 	% no special string spaces
	framexleftmargin=7mm, 
	tabsize=3,
	showtabs=false,
	frame=single, 
	rulesepcolor=\color{blue},
	numbers=left,
	linewidth=146mm,
	xleftmargin=8mm
}

\lstdefinelanguage{custom}
{
morekeywords= { abstract, event, new, struct,
                as, explicit, null, switch,
                base, extern, object, this,
                bool, false, operator, throw,
                break, finally, out, true,
                byte, fixed, override, try,
                case, float, params, typeof,
                catch, for, private, uint,
                char, foreach, protected, ulong,
                checked, goto, public, unchecked,
                class, if, readonly, unsafe,
                const, implicit, ref, ushort,
                continue, in, return, using,
                decimal, int, sbyte, virtual,
                default, interface, sealed, volatile,
                delegate, internal, short, void,
                do, is, sizeof, while,
                double, lock, stackalloc,
                else, long, static,
                enum, namespace, string, var, await, async, get, set },
morestring = *[d]{"},
stringstyle=\ttfamily\color{red}
}


\usepackage{textcomp} 			% Celsius-Darstellung
\usepackage{amssymb,amsfonts,amstext,amsmath}	% Mathematische Symbole
\usepackage[german, ruled, vlined]{algorithm2e}
\usepackage[a4paper]{geometry} % Andere Formatierung
\usepackage{bibgerm}
\usepackage{array}
\hyphenation{Ele-men-tar-ob-jek-te  ab-ge-tas-tet Aus-wer-tung House-holder-Matrix Le-ast-Squa-res-Al-go-ri-th-men} 		% Weitere Silbentrennung bei Bedarf angeben
\setlength{\textheight}{1.1\textheight}
\pagestyle{myheadings} 			% Erzeugt selbstdefinierte Kopfzeile
\makeindex 						% Index-Erstellung

\usepackage[official]{eurosym}
\usepackage{wrapfig}
%\usepackage{amsthm}
\usepackage{amsfonts}
\usepackage[mathscr]{euscript}

\usepackage{subfig}

\usepackage{url}
\def\UrlBreaks{\do\/\do-} 

\usepackage{epstopdf}

\usepackage{comment}
\usepackage{amsmath}
\usepackage{breqn}

\usepackage{blindtext}
\usepackage{scrextend}
\addtokomafont{labelinglabel}{\sffamily}

\lstset{basicstyle=\scriptsize\ttfamily,breaklines=true}

\lstdefinestyle{sharpc}{language=[Sharp]C, frame=lr, rulecolor=\color{black}}
\lstset{style=sharpc}

\usepackage{afterpage}

\usepackage{tabularx}
\usepackage{tabulary}

\usepackage[subfigure]{tocloft}
\tocloftpagestyle{empty}
%\setlength{\cftsubsecnumwidth}{4em}


%--------------------------------------------------------------------------
\begin{document}

\newpage
\thispagestyle{empty}


%------------------------- Titelblatt -------------------------------------
\title{Entwurf und Implementierung einer grafischen, domänenspezifischen Sprache zur Spezifizierung des Konversationsrouting in Contactcentern}
\subtitle{Design and Implementation of a Graphical DSL (Domain-Specific Language) for the Conversation Routing in Contact Centers}
%---- Die Art der Dokumentation kann hier ausgewählt werden---------------
%\project{Bachelor-Projektarbeit}
%\project{Bachelor-Abschlussarbeit}
%\project{Master-Projektstudium}
\project{Master-Abschlussarbeit}
%\project{Seminar zur Vorlesung ...}
%\project{Hausarbeit zur Vorlesung ...}
%--------------------------------------------------------------------------
\supervisor{Professor Dr. rer. nat. Rainer Oechsle} 		% Betreuer der Arbeit
\author{David Wichter} 							% Autor der Arbeit
\address{Trier,} 							% Im Zusammenhang mit dem Datum wird hinter dem Ort ein Komma angegeben
\submitdate{06.02.2018} 				% Abgabedatum
%\begingroup
%  \renewcommand{\thepage}{title}
%  \mytitlepage
%  \newpage
%\endgroup
\begingroup
  \renewcommand{\thepage}{Titel}
  \mytitlepage
  \newpage
\endgroup
%--------------------------------------------------------------------------

%-----------------------------------
% Sperrvermerk
%-----------------------------------
\afterpage{%
\thispagestyle{empty}

\section*{Sperrvermerk}
Die vorliegende Abschlussarbeit mit dem Titel ``Entwurf und Implementierung einer grafischen, domänenspezifischen Sprache zur Spezifizierung des Konversationsrouting in Contactcentern'' enthält unternehmensinterne Daten der Firma ilogixx. Daher ist sie nur zur Vorlage bei der Hochschule Trier sowie den Begutachtern der Arbeit bestimmt. Für die Öffentlichkeit und dritte Personen darf sie nicht zugänglich sein.

\par\medskip
\par\medskip

%\_\_\_\_\_\_\_\_\_\_\_\_\_\_\_\_\_\_\_\_\_ \hspace{1.5cm} \_\_\_\_\_\_\_\_\_\_\_\_\_\_\_\_\_\_\_\_\_\_\_\_ \\
%(Trier, den 06.02.2017)\hspace{1.5cm}
%(David Wichter)
\begin{minipage}[t]{3cm}
\rule{4cm}{0.5pt}
Datum
\end{minipage}
\hfill
\begin{minipage}[t]{9cm}
\rule{9cm}{0.5pt}
Unterschrift der Kandidatin / des Kandidaten
\end{minipage}

\newpage

}

%--------------------------------------------------------------------------

\frontmatter 
%--------------------------------------------------------------------------
\preface

%% deutsch
\paragraph*{}
Ich möchte mich bei der Firma ilogixx für das Thema und die Chance des vorliegenden Abschlussprojektes bedanken. Im Speziellen danke ich Christian Becker und Norbert Sondag, ohne die diese Arbeit in ihrer jetzigen Form nicht möglich gewesen wäre.				% Vorwort (optional)
\kurzfassung

%% deutsch
\paragraph*{}
In der vorliegenden Ausarbeitung wird der Entwurf und die anschließende Implementierung einer grafischen domänenspezifischen Sprache dokumentiert, mit der das Verhalten einer automatischen Kontaktverteilung für eingehende Anrufe in einem Contact Center programmiert werden kann. Dabei wird zuerst nach einer kurzen Einleitung und der Erläuterung der Motivation auf die Sprache und ihre Elemente eingegangen. Anschließend wird die Umsetzung ihrer Software-Komponenten besprochen. Diese bestehen aus einem grafischen Editor zum Erstellen von Modellen in der Sprache und einem Transformator, welcher ein so erstelltes Modell in ausführbaren CIL-Bytecode konvertiert. Die Arbeit schließt mit einer Beschreibung, wie das implementierte System getestet wurde, einer Erörterung der Performanz und Komplexität des geschriebenen Codes sowie einem Fazit, in dem ein Ausblick auf kommende Funktionen der Sprache gegeben wird. 			% Kurzfassung Deutsch/English
\newpage
\tableofcontents 						% Inhaltsverzeichnis
\newpage
\listoffigures 							% Abbildungsverzeichnis (optional)
%\listoftables 							% Tabellenverzeichnis (optional)
%--------------------------------------------------------------------------
\mainmatter                        		% Hauptteil (ab hier arab. Seitenzahlen)
%--------------------------------------------------------------------------
% Die Kapitel werden in separaten .tex-Dateien abgelegt und hier eingebunden.
\chapter{Einleitung}
\label{chap:Einleitung}

DUMMY CITATION DAMIT BIBTEX SEINE SCHEISS FRESSE HÄLT UND DEN SCHEISS BAUT: \cite{Coloouris:02}

\section{MyContactCenter}
In vielen Bereichen modernen Lebens hat sich seit dem Anstoß des digitalen Zeitalters ein signifikanter Wandel eingestellt [CITATION]. Das weite Feld  der zwischenmenschlichen Kommunikation zeigt dies wie kaum ein anderes auf. Auch an der Schnittstelle zwischen Privat- und Arbeitsleben zeigt sich diese Modernisierung und im Speziellen an der Kommunikation zwischen Unternehmen und ihren Kunden. Eine moderne Institution dieser Kommunikation ist das Contact Center. Hierbei handelt es sich um einen zentralen Anlaufpunkt für Kunden, an dem diese über verschiedene Kommunikationskanäle ein Anliegen anbringen können, welches von Agenten des Unternehmens bearbeitet wird. Diese Kommunikation ist jedoch nicht einseitig: Auch Contact Center nehmen Kontakt zur Kundschaft auf, um beispielsweise Marktforschung zu betreiben. 
\newline
Moderne Call Center setzen vermehrt auf die Möglichkeiten des Internets und im speziellen auf IP-Telephonie, um Ihre Anforderungen der Kommunikation zu erfüllen [CITATION]. Die von der Firma ilogixx hergestellte Software-Lösung MyContactCenter (im Folgenden mit MyCC abgekürzt) hat zum Ziel, Betreiber von Contact Center dabei zu unterstützen. Eine Kernaufgabe der Anwendung ist die Automatische Kontakt Verteilung (kurz ACD für automatic contact distribution). Das Produkt nimmt hierbei Anfragen auf unterschiedlichen Kommunikationskanälen entgegen, kategorisiert diese nach Sprache und Aufgabenbereich, und teilt sie freien Agenten zu, welche für diese Aufgabenbereiche beziehungsweise Sprache eingeteilt sind. Die Agenten interagieren nun über MyCC mit dem Anfragenden, bis die Kommunikation auf beiden Seiten beendet wird. Anschließend hat der Agent nun eine Nachbearbeitungszeit, in der das Anliegen des Anfragenden zu Ende gebracht werden kann. Danach ist der Agent wieder für die nächste Anfrage frei.
\newline
MyContactCenter ist eine verteilte Anwendung. Die Hauptkomponente ist der Server, welcher den Großteil der Verwaltungsaufgaben übernimmt. Er verwaltet eine Liste der angemeldeten Agenten, sowie deren aktuellen Kommunikationsstatus (frei, im Gespräch, etc.) und Nutzerdaten (Name, Sprache, Kompetenzbereich, etc.). Auch gehört das Kategorisieren und Zuordnen von Anfragen während der automatischen Kontakt Verteilung, und das weiterleiten dieser an den nächsten freien Agenten zu seinen Aufgaben. Konfiguriert wird der Server über einen Administrations-Client. Hier kann der Administrator des Contact Centers alle für den Betrieb benötigten Einstellungen vornehmen, wie zum Beispiel das Anlegen von neuen Sprachen und Aufgabenbereichen, oder das Verwalten von Agenten-Daten. Agenten besitzen eine eigene Client-Anwendung, die sich für die Dauer der Nutzungszeit am Server anmeldet.  Mit diesem Client kann der Agent vom Server erhaltene Anfragen bearbeiten, also zum Beispiel auf eine Email antworten oder sein IP-Telefon steuern. Dabei setzt MyCC auf bestehende Kommunikations-Infrastruktur auf. So wird beispielsweise die eigentliche IP-Telefonie nicht von MyCC implementiert, sondern von einer darunter liegenden Telefonanlage, mit der MyCC über eine Scripting-Schnittstelle kommuniziert.
\newline
Die Vermittlung einer Konversationsanfrage eines Kunden zu einem Agenten stellt also eine der Hauptaufgaben von MyContactCenter dar. Die vorliegende Ausarbeitung beschäftigt sich mit dem Entwurf und der Implementierung einer Erweiterung des Produktes, welche es den Betreibern des Contact Centers ermöglichen soll, noch mehr Kontrolle über die Behandlung der Konversationsanfrage zu erhalten. Mittels einer domänenspezifischen Sprache soll durch den Administrator das Konversationsrouting programmierbar sein, wie eine eingehende Anfrage vor und nach der Zustellung zum Agenten behandelt wird.  
















\section{Motivation}
TODO

\section{Domänenspezifische Sprachen}
TODO

\section{Benötigtes Vorwissen}
TODO

\subsection{.NET}
TODO 
 
\subsection{Roslyn}
TODO

\subsection{Asynchrone Methodenausführung in .NET}
TODO

\subsection{SIP}
TODO
\chapter{Entwurf}

\section{Anforderungen}
Einige Teilmenge der Anforderungen der domänenspezifischen Sprache wurden bereits in Kapitel \ref{sec:Motivation} in Form von User-Stories aufgeführt. Im Folgenden werden die Anforderungen in einer allgemeineren und umfassenderen Form wiedergegeben. Die zentrale Aussage die den Anforderungen zu Grunde liegt ist die zusammenfassende Formulierung des zu lösenden Problems: ``Benutzer von MyContactCenter brauchen eine Möglichkeit, das Routing von Konversationsanfragen für eine automatische Kontaktverteilung frei programmieren zu können''. Die Umsetzung dieser Vision ist das langfristige Ziel, sprengt allerdings den Rahmen einer Masterarbeit und damit den der vorliegenden Ausarbeitung. Zwar wurde beim Entwurf Acht darauf gegeben, allen Arten von Konversationsanfragen gerecht zu werden. Für die vorliegende Implementierung beschränkt sich die Ausarbeitung aber lediglich auf das Routing von eingehenden Telefonanrufen. Die relevante Aufgabe lautet also: ``Benutzer von MyContactCenter brauchen eine Möglichkeit, das Routing eingehender Telefonanrufe für eine automatische Kontaktverteilung frei programmieren zu können''.
\newline
Aus dieser Aufgabe ergeben sich folgende Anforderungen für die DSL:
\begin{itemize}
\item Die DSL muss eine Sammlung an Instruktionen bereit stellen, welche mit einem eingehenden Anruf interagieren
\item Instruktionen der DSL müssen sich in einer Reihenfolge arrangieren lassen, welche den zeitlichen Ablauf eines Anrufroutings spezifiziert. Ein solcher Ablauf muss Verzweigungen und Schleifen zulassen.
\item Die Sammlung an Instruktionen müssen folgende Interaktionen mit einem eingehenden Anruf möglich machen:
	\begin{itemize}
	\item Entgegen nehmen eines Anrufs
	\item Abspielen von Audiodateien 
	\item Abrufen von Eingaben des Anrufers über Mehrfrequenzwahlverfahren
	\item Ausführen von Benutzerscripten. Bei Bedarf soll ein Benutzer über die Programmiersprache C\# eigene Scripte programmieren können, welche dann im Routing ausgeführt werden.
	\item Einordnen von Anrufen in die Kategorien ``Sprache'' und ``Wissensbereich''. Die Einordnung eines Anrufes in diese Kategorien bestimmt, welche Agenten den Anruf entgegen nehmen können.
	\item Terminieren eines Anrufs
	\item Zustellung eines Anrufs an einen Agenten
	\end{itemize}
\item Ein Ablauf von Instruktionen muss mit einer grafischen Benutzeroberfläche arrangierbar sein.
\item Ein Ablauf von Instruktionen muss in ausführbaren Code transformiert werden können.
\item Der Ersteller eines Ablaufs von Instruktionen soll durch eine hohe Benutzerfreundlichkeit unterstützt und davor bewahrt werden, unnötige Fehler zu machen.  
\end{itemize}
Die obenstehende Liste stellt die Mindestanforderungen dar, die erfüllt sein müssen, damit ein Benutzer über die gewünschten Konfigurationsmöglichkeiten für eine Automatische Kontaktverteilung verfügt. Der im Folgenden Kapitel beschriebene Entwurf einer domänenspezifischen Sprache ist ein Versuch, ein System zu designen, welches die obenstehenden Anforderungen erfüllt. Zur Umsetzung des Entwurfs wurde ein grafischer Editor und ein Code-Generator umgesetzt, deren Implementierung näher in Kapitel \ref{chap:Implementierung} erläutert wird.

\section{Beschreibung der Sprache}
TODO

\section{Sprachelemente}
TODO

\subsection{Start}
TODO

\subsection{MediaPlayback}
TODO

\subsection{DTMF-Abfrage}
TODO

\subsection{Skills und Languages}
TODO

\subsection{Variablen}
TODO

\subsection{Verzweigungen}
TODO

\subsection{Scripte}
TODO

\subsection{Zustellung}
TODO

\subsection{Terminierung}
TODO

\section{Verarbeitungsschritte}
TODO

\chapter{Implementierung}
\label{chap:Implementierung}

\section{Editor}
\label{sec:Editor}
Der Editor ist die Schnittstelle der DSL mit einem menschlichen Benutzer. Hier wird ein DSL-Modell durch Interaktion mit der konkreten Syntax erstellt oder bearbeitet. Die konkrete Syntax nimmt die Form eines Graphen an, der in Kapitel \ref{chap:Einleitung} beschrieben ist. Die Aufgabe des Editors ist es, die konkrete Syntax in abstrakte Syntax umzuwandeln. Während der Benutzer über die grafische Benutzeroberfläche mit der konkreten Syntax interagiert, also Instruktionen hinzufügt und Verbindungen zieht, wird Editor-intern die abstrakte Syntax geformt, welche der Benutzer nicht sieht (vgl. [CITATION Voelter, DSL Engineering, Abs 3.4, PDF Seite 68]. Die abstrakte Syntax ist durch eine C\#-Datenstruktur repräsentiert, welche für die weiteren Verarbeitungsschritte gespeichert wird (siehe Abschnitt \ref{sec:Verarbeitungsschritte}). In der Literatur wird die Datenstruktur, welche die abstrakte Syntax beinhaltet, auch semantisches Modell (engl. semantic model) genannt, vgl. [CITATION Martin Fowler, Domain Specific Languages] und wird daher im Folgenden auch so referenziert. Der Editor bildet also die abstrakte Syntax auf das semantische Modell ab. Diese Abbildung ist bidirektional: Wird das semantische Modell manipuliert, spiegelt die grafische Benutzeroberfläche dies in der konkreten Syntax wider.
\newline
Der Editor ist als eine Windows Forms-Anwendung umgesetzt und implementiert das Model-View-Viewmodel-Entwicklungsmuster (kurz MVVM). Ähnlich wie bei verwandten Entwurfsmustern wie zum Beispiel Model-View-Controller (MVC) geht es bei MVVM darum, die Darstellung einer Anwendung von ihrer Logik zu trennen. Dies geschieht durch eine Einteilung in drei verschiedene Schichten: Dem View, dem Viewmodel und dem Model. Die View-Schicht dient zur Darstellung der Anwendung und umfasst die Komponenten der grafischen Benutzeroberfläche. Sie erhält die nötigen Daten zur Darstellung vom Viewmodel über sogenannte Datenbindung (engl. Data Binding). Dabei werden Daten über ein Framework an einander gebunden und werden so vom System bei jeder Zustandsänderung synchron gehalten. Ändern sich also Daten im Viewmodel wird dies unverzüglich im View angezeigt. Das Viewmodel kümmert sich um die Präsentationslogik, das heißt es ist dafür zuständig, dem View Daten zur Verfügung zu stellen, welche über Datenbindungen angezeigt werden sollen. Dafür hat das Viewmodel Zugriff auf das Model, über dass die Daten zur Darstellung abgerufen werden. Im Model werden auch die Geschäftslogik sowie Speicher- und Lade-Aktionen ausgeführt. In .NET wird MVVM vor allem im Zuge der Windows Presentation Foundation eingesetzt, kann aber auch durch das Devexpress-Framework in Windows Forms-Anwendungen benutzt werden.
\newline
Im grafischen Editor der DSL werden die drei Schichten von MVVM durch verschiedene C\#-Klassen abgebildet, welche im Folgenden erläutert werden.

\subsection{Die Model-Schicht: Flow, Node, Input, Output, Connector, Variable, Function}
\label{subsec:Die Model-Schicht}
Die Model-Schicht beinhaltet die C\#-Datenstruktur, welche auch als semantisches Modell bezeichnet wird: Also das Objekt, welches in der weiteren Verarbeitung die abstrakte Syntax beschreibt. Das semantische Modell selbst weist kein programmitisches Verhalten auf, sondern ist eine reine Datenstruktur. Es besteht aus Instanzen von Klassen, welche die einzelnen Elemente der DSL modellieren und im Namespace ilogixx.ConversationFlow.Core definiert sind. Die Klassenstruktur der Model-Schicht ist in Abbildung [IMAGE UML MODEL] als UML-Diagramm aufgeführt und wird im Folgenden erläutert.
\newline
Instruktionen werden durch die abstrakte Klasse Node modelliert. Node besitzt eine Liste von Referenzen auf Output-Objekte, welche die Ausgänge einer Intruktion modellieren. Node hat ebenfalls eine Referenz auf ein Objekt vom Typ Input, welches den Eingang eines Knotens symbolisiert. Jede Art von Instruktion hat ihre eigene Subklasse, welche die Instruktionen aus Abschnitt \ref{sec:Sprachelemente} abbilden. Pro Ausgang einer Instruktion fügt der Konstruktor des Subtypen eine Referenz vom Typen Output der Referenzliste der Basisklasse hinzu. Im Konstruktor wird ebenfalls der Input spezifiziert: Hat die Instruktion einen Eingang, wird die Input-Referenz instanziert, andernfalls bleibt sie null. Viele Instruktionen haben die gleiche Anzahl an Ausgängen und einen Eingang. Daher existieren Subtypen, welche diese Initialisierungsarbeit übernehmen: Inputnode, für Instruktionen die einen Eingang haben, SingleOutputNode für Instruktionen die einen einzelnen Ausgang haben, und SingleThroughputNode, für Instruktionen die einen Ein- und einen einzelnen Ausgang haben. Alle elf Subtypen für die Instruktionen erben von diesen drei Klassen und passen dort wo nötig ihre Konfiguration an. Beispielsweise erbt DTMFCharacterNode von InputNode, und fügt der Output-Referenzliste dreizehn neue Outputs hinzu.
\newline
Die Klasse Input ist simpel: Sie zeigt über eine Referenz vom Typen Node auf ihre besitzende Instruktion. Die Klasse Output hingegen besitzt einige mehrere Daten. Sie speichert den Bezeichner des Ausgangs als String, einen weiteren String namens DisplayString, welcher vom Benutzer editiert werden kann und auf der Benutzeroberfläche angezeigt wird, einen Boolean namens Visible, welcher bestimmt ob der Ausgang in dem Rechteck der beinhaltenden Instruktion angezeigt wird, und eine Referenz vom Typ Input auf einen Eingang, falls der Ausgang mit einem solchen verbunden sein sollte. Durch letztere Input-Referenz sind die Verbindungen im Konversationsrouting zwischen den Instruktionen implizit gegeben. Dennoch existiert eine weitere Klasse namens Connector, welche eine Verbindung explizit definiert. Sie besitzt Referenzen auf die Input- und die Output-Instanz, welche verbunden sind.
\newline
Benutzerdefinierte Variablen und Funktionen sind über eigene Klassen abgebildet: Variable und Function. Variable besitzen einen Namen vom Typen String und einen Typen, der über einen Enum namens VariableType abgebildet ist, welcher die in \ref{subsec:Variablen} aufgezählten Typen beinhaltet. Function speichert ebenfalls einen Namen vom Typen String und einen Rückgabetyp, welcher über einen Enum mit Namen FunctionReturnType abgebildet ist. Zusätzlich wird der Funktionskörper mit Namen Body als String gespeichert. Parameter einer Funktion werden über eine Liste von Referenzen auf Objekte des Type Parameter modelliert. Ein Parameter besitzt, ähnlich wie die Klasse Variable, einen Namen als String und einen Typen als VariableType. Die Klasse Flow vereint alle oben genannten Klassen: Sie beinhaltet Listen mit Referenzen auf alle Nodes, Connectors, Functions und Variables. 

\subsection{Die Viewmodel-Schicht: FlowDiagramViewmodel, CodeEditorViewModel, FunctionCollectionEditorViewModel}
\label{subsec:Die Viewmodel-Schicht}
In der Viewmodel-Schicht werden die Klasseninstanzen der Model-Schicht so aufbereitet, dass die View-Schicht sie per Datenbindung repräsentieren kann. Die Viewmodel-Schicht reagiert auch auf Eingaben des Benutzers und manipuliert das zu Grunde liegende Model nach dessen Wünschen. 
\newline
Der Hauptbestandteil dieser Schicht ist die Klasse FlowDiagramViewmodel. Diese besitzt als Member-Variablen Listen auf Referenzen der Model-Komponenten Node, Connector, Variable und Function. Diese Listen sind vom .NET-Typ ObservableCollection, welche ein Event auslösen, falls der Liste Objekte hinzugefügt oder entfernt werden. Diese Listen sind per Devexpress-Datenbindung an die View-Schicht gekoppelt. Das heißt, wenn auf der View-Schicht Elemente des Models wie neue Node- oder Connector-Objekte hinzugefügt werden, aktualisieren sich die entsprechenden Listen im FlowDiagramViewmodel. Zusätzlich werden dadurch Events ausgelöst, auf die FlowDiagramViewModel reagiert. Dies wird dazu genutzt, die Referenzen zwischen den Klassen Input und Output aktuell zu halten: Wird ein Connector der Liste hinzugefügt, wird die Input-Referenz des im Connector referenzierten Outputs auf den Input gesetzt, der ebenfalls im Connector verlinkt ist. Der Vorgang wir rückgängig gemacht, wenn ein Connector entfernt wird. FlowDiagramViewModel übernimmt mittels der beiden Kommandos WriteToDisk und ReadFromDisk auch das Speichern und Laden von Konversationsroutings. Dafür wird mit den Listen der Node-, Connector-, Variable- und Function-Instanzen ein Flow-Objekt instanziert und mit Protobuf serialisiert. Pfade für die zu schreibenden beziehungsweise lesenden Dateien werden über Dialoge vom Benutzer abgefragt. Die Dialoge werden über vorgefertigte Serviceklassen angezeigt, welche das MVVM-Framework von Devexpress per Dependency Injection zur Verfügung stellt. 
\newline
FlowDiagramViewModel befindet sich hinter der Ansicht, in der der Benutzer sein Konversationsrouting modelliert. Einige Instruktionen benötigen jedoch die Möglichkeit, Code einzugeben. Zu diesem Zweck existiert ein weiteres ViewModel, welches dem eingebauten Code-Editor zu Grunde liegt. Das CodeEditorViewModel beinhaltet einen String, welcher den Code enthält. Dieser ist per Datenbindung an den Text des Code-Editors gebunden. Zusätzlich enthält das CodeEditorViewModel entweder eine Referenz auf die zu bearbeitende ScriptNode, oder die zu bearbeitende Function, welche den zu bearbeitenden Code enthält. Eine Aufgabe des CodeEditorViewmodels ist es den im Editor eingegebenen Code synchron mit dem code in der jeweiligen ScriptNode oder Function zu halten. 
\newline
Damit der Benutzer die selbst definierten Funktionen verwalten kann, existiert zusätzlich ein Funktions-Editor, in dem neue Funktionen angelegt, gelöscht oder bearbeitet werden können. Für diesen Editor existiert ebenfalls ein eigenes ViewModel: Das FunctionCollectionViewModel. Es beinhaltet eine Referenz auf die Liste der Funktionen, die sich im FlowDiagramViewModel befindet und wendet das Ändern, Löschen oder Hinzufügen von Funktionen auf diese Liste an.


\subsection{Die View-Schicht: FlowDiagramView, FlowDiagramControl, FlowDiagramItem}
Die View-Schicht übernimmt alle Elemente der grafischen Benutzeroberfläche, die der Benutzer unmittelbar sieht und mit denen er interagiert. Das Hauptelement der View-Schicht ist die Klasse FlowDiagramView, welche von der Framework-Klasse UserControl erbt. FlowDiagramView umfasst alle Windows Forms-Controls, die zum Darstellen des Editors benötigt werden und übernimmt zusätzlich die Konfiguration des MVVM-Frameworks während der Initialisierungsphase, in der die Benutzeroberfläche geladen wird.
\newline  
Das wichtigste Element der Benutzeroberfläche ist das Control FlowDiagramControl, welches  die abstrakte Syntax des Konversationsroutings darstellt. Es basiert auf der Devexpress-Klasse DiagramControl, die es Benutzern ermöglicht, Diagramme mit geometrischen Formen und Verbindungen zu gestalten. FlowDiagramControl ist von von dieser Klasse abgeleitet und erweitert sie um die Fähigkeit, ein Modell der DSL per Datenbindung darzustellen. Dafür sind in FlowDiagramControl Listen vom Typ ObservableCollection als Membervariablen angelegt, welche die Elemente des aktuellen DSL-Modells enthalten. Diese Listen existieren analog zur Klasse FlowDiagramViewModel aus Abschnitt \ref{subsec:Die Viewmodel-Schicht}. Durch die Datenbindung zwischen FlowDiagramControl und FlowDiagramViewModel werden die Inhalte aller Listen synchron gehalten. Die Aufgabe von FlowDiagramControl ist es, auf Interaktionen des Benutzers so zu reagieren, dass die Listen entsprechend aktualisiert werden. Zu diesem Zweck ist diese Klasse in einigen Aspekten gegenüber DiagramControl wie folgt angepasst und spezialisiert.
\newline
Für ein Devexpress-DiagramControl kann eine Auswahl von sogenannten Stencils, also für das Diagramm verfügbare Formen dem Control hinzugefügt werden. Jedes Stencil verfügt über ein Objekt der Klasse FactoryItemTool, welches eine Instanz vom Interface-Typ IDiagramItem erzeugt. Wird ein Stencil aus der Toolbox in das Diagram per Drag and Drop eingefügt, erzeugt das zugehörige FactoryItemTool ein entsprechendes Objekt, welches IDiagramItem implementiert. IDiagramItem stellt Methoden zu Verfügung, die DiagramControl benötigt um Objekte grafisch darzustellen. FlowDiagramControl ist so konfiguriert, dass jeder Node-Subtyp, also jede Instruktion der DSL, als eigenes Stencil dem Control hinzugefügt ist. Die Instanzen von FactoryItemTool für jedes dieser Stencil erzeugen ein Objekt vom Typ FlowDiagramItem. FlowDiagramItem ist die visuelle Repräsentation einer Konversationsrouting-Instruktion und erbt von der Devexpress-Klasse DiagramContainer, welche IDiagramItem implementiert. FlowDiagramItem besitzt eine Referenz auf ein Objekt vom Typ Node. Im Konstruktor von FlowDiagramItem wird das visuelle Erscheinungsbild einer einzelnen Instruktion initialisiert: Dabei werden Klassen vom Devexpress-Typ DiagramShape zu einer Liste hinzugefügt, welche dann vom Framework gezeichnet werden und verschiedene geometrische Formen annehmen können. FlowDiagramItem fügt der Liste von DiagramShapes ein blaues Rechteck für den Körper, ein weißes Rechteck für die Instruktionsüberschrift sowie kleinere rote und grüne Rechtecke für Instruktionsaus- und Eingänge hinzu. Wieviele Aus- und Eingänge die jeweilige zu visualisierende Instruktion besitzt sowie die Überschrift wird aus dem Subtyp der Node-Referenz ermittelt.
\newline
Wird nun vom Benutzer ein Stencil aus der Toolbox ausgewählt und in das Modellierungsfenster gezogen, erzeugt die zugehörige Instanz von FactoryItemTool ein FlowDiagramItem. Dieses FlowDiagramItem wird mit einer Referenz auf eine neu erstellte Instanz des Node-Subtyps der ausgewählten Instruktion initialisiert. DiagramControl löst dann ein Event aus, das signalisiert, dass sich die Liste an Diagramm-Elementen verändert hat. FlowDiagramControl reagiert auf dieses Event seiner Basisklasse, indem es die Node-Referenz aus dem hinzugefügten FlowDiagramItem zu der Liste der Nodes hinzufügt, die per Datenbindung an das FlowDiagramViewModel gebunden sind. Umgekehrt reagiert FlowDiagramControl auch auf Events, die ausgelöst werden, wenn die Liste der Nodes durch das ViewModel verändert wird. In diesem Fall wird ein neues FlowDiagramItem mit einer Referenz auf die hinzugefügte Node erzeugt, sodass diese Node auf der Benutzeroberfläche zu sehen ist. Analog verhält sich das System wenn Nodes oder FlowDiagramItems gelöscht werden, nur dass in diesem Fall Elemente entfernt werden.
\newline
Um die Verbindungen zwischen Instruktionen zu modellieren, wird ähnlich vorgegangen. FlowDiagramControl überschreibt die geerbte Methode CreateConnector, welche vom Framework aufgerufen wird, wenn der Benutzer eine Verbindung zwischen Diagrammelementen zieht. Die Methode liefert eine Instanz von FlowDiagramConnector zurück, in der eine Referenz auf einen Connector der Model-Schicht hinterlegt ist. Wird ein FlowDiagramConnector dem Diagram hinzugefügt, wird über ein entsprechendes Event reagiert indem ein neuer Connector der per Datenbindung synchronisierten Liste hinzugefügt wird. In FlowDiagramViewModel wird dementsprechend ebenfalls ein neuer Connector hinzugefügt und die Input- und Output-Referenzen der im Connector betroffenen Ein- und Ausgänge werden aktualisiert. Wird ein Connector der Modelschicht hinzugefügt, läuft der Prozess rückwärts ab. Das heißt, über die synchronisierte Liste der Connector-Instanzen wird im ebenfalls im FlowDiagramControl ein Event ausgelöst, so dass das Control einen neuen FlowDiagramConnector dem Diagramm hinzufügen kann. Beim Entfernen von Connector-Instanzen wird er Prozess gleich behandelt, nur dass Referenzen entsprechend entfernt werden.      


\section{Transformation}
\label{sec:Transformation}
Die Transformation ist der Schritt in der Verarbeitungskette, in der aus einem semantischen Modell, welches in der Form der C\#-Datenstruktur aus Abschnitt \ref{subsec:Die Model-Schicht} vorliegt, eine MSIL-Syntax generiert wird. Diese kann anschließend compiliert und ausgeführt werden. Ein einzelnes Modell wird auf die Syntax einer Klasse abgebildet, welche das gewünschte Verhalten des Modells implementiert. Diese Klasse erbt von einer abstrakten Basisklasse mit dem Namen ACDCallRoutingBehaviorBase, welche die abstrakte Methode StartAsync zur Verfügung stellt. Die vom Transformationsalgorithmus generierte Klasse implementiert StartAsync so, dass beim Aufruf das vom Benutzer spezifizierte Konversationsrouting abgespielt wird.
\newline 
Die Instruktionen des Konversationsroutings werden als Syntax für Membermethoden der Klasse abgebildet. In der generierten Klasse existiert für jede Instruktion eine eigene Methode, deren Syntax das gewünschte Verhalten der Instruktion umsetzt. Der Bezeichner der Methode trägt den Namen der Instruktion gefolgt von einer eindeutigen Identifikationsnummer, sodass mehrere Instruktionen des gleichen Typs bei der Generierung der zugehörigen Membermethoden nicht kollidieren. Manche Instruktionen müssen mit der zu routenden Konversation interagieren. Zu diesem Zweck steht der generierten Klasse eine API zur Verfügung, welche aus einer Instanz eines Interfaces namens IRoutedAcdCall besteht. Soll beispielsweise im Zuge einer Media Playback-Instruktion eine Audio-Datei abgespielt werden, kann Syntax generiert werden, welche die Methode PlayAudioAsync des Interfaces IRoutedAcdCall aufruft. 
\newline
Die Reihenfolge von Instruktionen in einem Konversationsrouting wird realisiert, indem generierte Methoden, welche Instruktionen abbilden, die Methoden ihrer Nachfolgeinstruktionen aufrufen. Angenommen, eine Branch-Instruktionen hat zwei Ausgänge: Der ''True``-Ausgang zeigt auf eine Media Playback-Instruktion während der ''False``-Ausgang auf die Terminate-Instruktion verweist. Aus diesen drei Instruktionen wird Syntax für drei Membermethoden generiert: Die Methode der Branch-Anweisung wertet den per Parameter übergebenen booleschen Ausdruck aus und ruft im True-Fall die Methode der Media Playback-Instruktion auf. Andernfalls wird die Methode der Terminate-Anweisung aufgerufen. Da die Terminate-Anweisung keine Ausgänge hat, erreicht das Konversationsrouting dort ihr Ende. Da es sich bei dem semantischen Modell um einen Graphen und keinen Baum handelt, kann es in Modellen zu Zyklen kommen. Solche potentiell endlose Konversationsroutings stellen ein Problem dar, denn sie führen zu einer endlosen Rekursion von Methodenaufrufen innerhalb der generierten Klasse. Das Problem wird gelöst indem eine Instruktion, die eine solche Rekursion starten würde, die nächste Instruktion in einem .NET-Task auf einem neuen Thread aufruft. Der aktuelle  Thread terminiert an der Stelle, an der der neue gestartet wird und es wird eine endlose Rekursion vermieden.
\newline
Ein Modell kann vom Benutzer geschriebenen C\#-Code enthalten. Dieser muss in die generierte Klasse integriert werden, um ausgeführt zu werden. Benutzerdefinierten Code jedoch einfach in die Syntax für Membermethoden, die aus Instruktionen generiert werden,  einzufügen birgt Risiken. Denn so hat der Benutzer in seinem Code Zugriff auf alle Elemente der generierten Klasse, was zu Problemen führen kann, wenn der Methoden aufruft, die der generierten Klasse vorbehalten sind. Code des Benutzers wird stattdessen in einer verschachtelten privaten Klasse gekapselt. 

\subsection{Beispiel}
TODO

\subsection{Umsetzung}
Die Grundidee der Transformationsumsetzung orientiert sich an dem Verfahren, das in [CITATION Voelter S. 272] als ''klassische Modell-Transformierung`` (engl. classical model transformation) bezeichnet wird: Das semantische Modell wird als Graph traversiert, während mit der Roslyn-API anhand des aktuell besuchten Knotens passende Syntax generiert wird.

\subsubsection{Programmablauf}
TODO

\subsubsection{User-Code}
TODO

\subsubsection{Nebenläufigkeit}
TODO

\subsection{Einbindung von generiertem Code}
TODO

\section{Modell-Validierung} 
TODO
\chapter{Evaluierung}

\section{Test}
Die Verifikation der DSL und ihrer Komponenten wird auf zwei Arten durchgeführt: Zum einen mit automatisierten Modultests und zum anderen mit manuellen Integrationstests. Erstere zielen darauf ab, die Funktionsweise von Editor, Transformator und der Modell-Validierung einzeln zu testen und zu verifizieren, dass diese ihre Anforderungen erfüllen. Letztere sollen das Zusammenspiel aller Module der DSL prüfen, indem der Erstellungsprozess für ein Konversationsrouting von der Modellierung bis zur Ausführung nachvollzogen wird.

\subsection{Automatisierte Tests}
Für die automatisierten Tests der DSL-Komponenten kommen Modultests, auch Unit Tests genannt, zum Einstaz. Befürworter der Unit Testing-Praktik befürworten an dieser Technik neben der erhöhten Code-Qualität unter anderem die frühe Ermittlung von Fehlerzuständen (vgl. \cite{Novoseltseva:17}). In der vorliegenden Ausarbeitung wird das Unit Test-Verfahren mit Hilfe des .NET-Test-Frameworks XUnit umgesetzt. Dabei werden Methoden mit einem Attribut als Test-Methoden gekennzeichnet, welche dann von XUnit ausgeführt und ausgewertet werden. Die Testmethoden folgen dem Arrange-Act-Assert-Entwicklungsmuster. Nach diesem Muster ist eine Methode in drei Phasen aufgeteilt: In der Arrange-Phase werden die für den Test nötigen Vorbereitungen wie Objekt-Initialisierungen etc. durchgeführt. Anschließend folgt die Act-Phase, in der die zu testende Aktion durchgeführt wird. In der Assert-Phase wird das Testergebnis mit Assertions überprüft. Schlägt keine der Assertions fehl, gilt ein Testfall als bestanden.

\subsubsection{Editor}
In der Editor-Komponente werden zwei Klassen getestet: \texttt{FlowDiagramViewModel} und \texttt{FlowDiagramControl}. Die beiden Komponenten sind vor allem über die Datenbindung des Devexpress-Frameworks miteinander verbunden. Da die Datenbindung jedoch eine Funktion des Devexpress-Framework ist und sich das Testen dieser außerhalb des Zuständigkeitsbereichs der vorliegenden Ausarbeitung befindet, werden die beiden Klassen unabhängig voneinander getestet. Für \texttt{FlowDiagramViewModel} ist das Verhalten bei Veränderungen in der Model-Schicht interessant, vor allem wenn es um das Hinzufügen oder Entfernen von Verbindungen geht. Folgende Aspekte der Klasse \texttt{FlowDiagramViewModel} werden unter anderem in den vorliegenden Unit Tests verifiziert: 

\begin{itemize}
\item Das Speichern von \texttt{Flow}-Instanzen im Dateisystem funktioniert
\item Das Laden von \texttt{Flow}-Instanzen aus dem Dateisystem funktioniert
\item Beim Hinzufügen von \texttt{Connector}-Instanzen werden die zugehörigen \texttt{Input}-Referenzen in betroffenen Outputs korrekt gesetzt
\item Beim Entfernen von \texttt{Connector}-Instanzen werden die zugehörigen \texttt{Input}-Referenzen in betroffenen \texttt{Output}-Referenzen auf null gesetzt
\item Beim Setzen einer \texttt{Input}-Referenz in einer \texttt{Output}-Instanz wird eine entsprechende \texttt{Connector}-Instanz hinzugefügt
\item Beim Setzen einer \texttt{Input}-Referenz auf null in einer \texttt{Output}-Instanz wird die entsprechende \texttt{Connector}-Instanz entfernt
\item Beim Verändern einer \texttt{Input}-Referenz in einem Output auf eine andere \texttt{Input}-Instanz wird die entsprechende \texttt{Connector}-Instanz aktualisiert 
\end{itemize}

Auch bei der Klasse \texttt{FlowDiagramControl} fokussieren sich die Testfälle auf das Verhalten im Fall einer Veränderung der Daten, die per Datenbindung zur Verfügung stehen. Die Hauptaufgabe von \texttt{FlowDiagramControl} ist es, diese Daten auf Instanzen der Klasse \texttt{FlowDiagramItem} abzubilden. Da die eigentliche Darstellung der \texttt{FlowDiagramItem}-Instanzen von der Basisklasse und damit vom Devexpress-Framework übernommen wird, wird dies nicht getestet. Stattdessen prüfen die Testfälle, ob es für jede \texttt{Connector}- und jede \texttt{Node}-Subtyp-Instanz eine entsprechende \texttt{DiagramItem}-Instanz gibt. Die Testfälle verifizieren daher folgende Anforderungen:

\begin{itemize}
\item Das Hinzufügen einer \texttt{Node}-Subtyp-Instanz fügt dem Diagramm eine entsprechende \texttt{Flow}\-Diagram\-Item-In\-stanz hinzu
\item Das Entfernen einer \texttt{Node}-Subtyp-Instanz entfernt auch die entsprechende \texttt{Flow}\-Diagram\-Item-In\-stanz
\item Das Hinzufügen einer \texttt{FlowDiagramItem}-Instanz fügt eine entsprechende \texttt{Node}-Subtyp-Instanz hinzu
\item Das Entfernen einer \texttt{FlowDiagramItem}-Instanz entfernt eine entsprechende \texttt{Node}-Subtyp-Instanz
\item Das Hinzufügen einer \texttt{Connector}-Instanz fügt dem Diagramm eine entsprechende \texttt{Flow}\-Dia\-gram\-Con\-nec\-tor-In\-stanz hinzu
\item Das Entfernen einer \texttt{Connector}-Instanz entfernt auch die entsprechende \texttt{Flow}\-Dia\-gram\-Con\-nec\-tor-In\-stanz
\item Das Hinzufügen einer \texttt{FlowDiagramConnector}-Instanz fügt auch eine entsprechende \texttt{Connector}-Instanz hinzu
\item Das Entfernen einer \texttt{FlowDiagramConnector}-Instanz entfernt auch die entsprechende \texttt{Connector}-Instanz
\item Das Ändern der Endpunkte einer \texttt{FlowDiagramConnector}-Instanz passt auch die entsprechende \texttt{Connector}-Instanz an
\end{itemize} 

\subsubsection{Transformator}  
In den Tests für die Transformator-Komponenten werden die wichtigsten Klassen für die Bytecode-Generierung verifiziert. Dabei ist der Zuständigkeitsbereich einer Klasse zu beachten, und dementsprechend zu testen. Es macht zum Beispiel wenig Sinn, die Details der Syntax der generierten Klassen in den Testfällen für \texttt{ConversationRoutingBehaviorGenerator} zu verifizieren, da sich diese Klasse mit der Kompilierung der Syntax und nicht deren Zusammensetzung beschäftigt. Die Syntax der generierten Klassen wird in den Modulen \texttt{FlowClassSyntaxBuilder} und \texttt{UserCodeClassSyntaxBuilder} erstellt und sollte dementsprechend in den dort zugehörigen Testfällen verifiziert werden. Auf diese Weise wird eine unnötige mehrfache Testüberdeckung vermieden. Vor diesem Hintergrund werden für die Klasse \texttt{ConversationRoutingBehaviorGenerator} folgende Aspekte mittels Unit Tests überprüft: 

\begin{itemize}
\item Im Falle einer übergebenen korrekten \texttt{Flow}-Instanz wird die CIL-Syntax erfolgreich erstellt, kompiliert und es wird eine In-Memory-Assembly zurückgegeben, in der die erwartete Konversationsrouting-Klasse definiert ist
\item Im Falle einer übergebenen \texttt{Flow}-Instanz, in der keine Start-Instruktion vorhanden ist, wird eine Exception geworfen
\item Im Falle einer fehlgeschlagenen Kompilierung der generierten Syntax wird eine Exception geworfen, in der detaillierte Fehlermeldungen enthalten sind
\item Im Falle einer übergebenen korrekten \texttt{Flow}-Instanz wird von Roslyn ein semantisches Modell der CIL-Syntax zurückgegeben
\end{itemize} 

Die nächst niedrigere Stufe bei der Transformation ist die Erstellung des CIL-Syntaxbaums. Hierfür wird die Klasse \texttt{ConversationRoutingSyntaxTreeBuilder} folgendermaßen getestet:

\begin{itemize}
\item Im Falle einer übergebenen Start-Instruktion wird erfolgreich ein CIL-Syntaxbaum generiert, welcher auf der höchsten Ebene die Syntax für die erwartete Klassendeklarationen im erwarteten Namespace enthält
\item Im Falle von ungenügenden übergebenen Parametern wird eine Exception geworfen
\end{itemize}

Der Syntaxbaum besteht zum größten Teil aus der Deklaration der Klasse, die das Konversationsrouting umsetzt. Die Syntax dieser Klasse wird von \texttt{FlowClassSyntaxBuilder} erstellt und im Zuge der zugehörigen Testfälle verifiziert:

\begin{itemize}
\item Im Falle einer übergebenen Start-Instruktion eines korrekten DSL-Modells in Form einer \texttt{StartNode}-Instanz wird die Syntax für eine Klasse zurückgegeben, die von \texttt{ACDCallRoutingBehaviorBase} erbt, und für jede im Modell enthaltene Instruktion eine Membermethode beinhaltet. Zusätzlich existiert eine geschachtelte private Klassendeklaration für den benutzerspezifischen Code, welche auch als Membervariable in der umfassenden Klasse vertreten ist
\item Im Falle von fehlerhaften übergebenen Parametern wird eine Exception geworfen.
\end{itemize}

\texttt{UserCodeClassSyntaxBuilder}, welche für die Generierung der geschachtelten Klassendeklaration zuständig ist, wird auf folgende Anforderungen getestet:

\begin{itemize}
\item Im Falle einer übergebenen Start-Instruktion eines korrekten DSL-Modells in Form einer \texttt{StartNode}-Instanz wird die Syntax für eine Klasse zurückgegeben, in der alle Instruktionen des Modells, die benutzerspezifischen Code enthalten, mit einer Membermethode vertreten sind. Diese Membermethoden führen den vom Benutzer geschriebenen Code aus. Zusätzlich existieren für benutzerdefinierte  Funktionen entsprechende Membermethoden und für benutzerdefinierte Variablen entsprechende Membervariablen. Die Klasse enthält weiterhin zwei Properties vom Typ String mit den Namen \texttt{Skill} und \texttt{Language}.
\item Im Falle von fehlerhaften übergebenen Parametern wird eine Exception geworfen.
\end{itemize}

Ob die Syntax für generierte Methoden, die den Ablauf des Routings steuern, die richtige Struktur aufweist, wird in den Testfällen für die Klasse \texttt{MethodFlowSyntaxBuilder} wie folgt getestet:

\begin{itemize}
\item Wird eine Instruktion in Form einer \texttt{Node}-Subtyp-Instanz übergeben, wird eine Liste an Methodendeklarationen zusammengestellt, in der alle von der übergebenen \texttt{Node}-Subtyp-Instanz erreichbaren Instruktionen abgebildet sind. In der Syntax für jeden Methodenkörper existieren die Methodenaufrufe für die im Modell jeweils nachfolgende Instruktion.
\item Werden unzureichende Parameter übergeben, wird eine Exception geworfen
\end{itemize}

Die Methoden-Deklarationen für die Klasse \texttt{UserCode} werden von \texttt{User\-Code\-Meth\-od\-Syn\-tax\-Buil\-der} umgesetzt und folgendermaßen getestet:

\begin{itemize}
\item Wird eine Instruktion in Form einer \texttt{Node}-Subtyp-Instanz übergeben, wird eine Liste an Methodendeklarationen zusammengestellt, in der nur die Instruktionen abgebildet sind, in denen vom Benutzer geschriebener Code zum Einsatz kommt.
\item Werden unzureichende Parameter übergeben, wird eine Exception geworfen
\end{itemize}

Zuletzt wird noch getestet, ob der Inhalt der generierten Methoden den Anforderungen entspricht. Dafür werden alle von \texttt{NodeSyntaxBuilder} abgeleiteten Typen getestet, da diese die instruktionsspezifische Syntax  für eine Methode generieren:

\begin{itemize}
\item Jeder Subtyp von \texttt{NodeSyntaxBuilder} erstellt für die ihm übergebene Instanz des zugehörigen \texttt{Node}-Subtyps die spezifische Syntax, die notwendig ist, um diese Instruktion umzusetzen. Die Syntax wird in Form einer Liste von \texttt{SyntaxNode}-Instanzen zurückgeliefert 
\item Werden unzureichende Parameter übergeben, wie zum Beispiel eine Instanz eines falschen \texttt{Node}-Subtyps, wird eine Exception geworfen
\end{itemize}

\subsubsection{Modell-Validierung}  
Die Modell-Validierung stellt eine eigene Komponente dar und wird daher auch in einer separaten Sammlung an Unit-Tests verfiziert. Im Fokus stehen vor allem die ConstraintEvaluators, welche auf ihre Funktionstüchtigkeit geprüft werden. Im Detail werden folgende Tests durchgeführt:

\begin{itemize}
\item Wird ein DSL-Modell ohne Start-Instruktion validiert, gibt \texttt{StartNodeExistsEvaluator} eine entsprechende Diagnostik aus
\item Wird ein DSL-Modell mit mehr als einer Start-Instruktion validiert, gibt \texttt{Start\-Node\-Ex\-ists\-E\-val\-u\-a\-tor} eine entsprechende Diagnostik zurück
\item Beinhaltet ein DSL-Modell eine Start-Instruktion, gibt \texttt{StartNodeExists} keine Diagnostik zurück
\item Für jede unverbundene Instruktion in einem DSL-Modell gibt \texttt{No\-Un\-con\-nect\-ed\-Node\-Ex\-ists} eine Diagnostik zurück, die auf die entsprechende Instruktion verweist
\item Für jeden ungültigen Instruktionsparameter gibt \texttt{Node\-Pa\-ra\-me\-ters\-Are\-Val\-id\-E\-val\-u\-a\-tor} eine Diagnostik mit Referenz auf die betroffene Instruktion zurück
\item Für jeden Compiler-Fehler in benutzerdefiniertem Code gibt \texttt{U\-ser\-Code\-Com\-piles\-E\-val\-u\-a\-tor} eine Diagnostik mit einer Referenz auf die betroffene Instruktion oder Funktion zurück
\item Jeder von \texttt{U\-ser\-Code\-Com\-piles\-E\-val\-u\-a\-tor} bemängelte Compiler-Fehler besitzt eine korrekte Stellenangabe, die anzeigt, wo im Benutzer-Code der Fehler liegt
\end{itemize} 

Bei allen Tests, die als Parameter eine Flow-Instanz entgegennehmen, wird das Modell aus Abb. \ref{fig:TestRouting} verwendet. 

\subsection{Manuelle Tests}
Bei den manuellen Tests geht es darum, das korrekte Zusammenwirken der einzelnen Komponenten zu testen. Daher werden alle Schritte, die für die Inbetriebnahme eines Konversationsroutings von Nöten sind, ausgeführt und die korrekte Funktionsweise anschließend verifiziert. Zu diesem Zweck werden folgende Schritte in der beschriebenen Reihenfolge ausgeführt:
\begin{description}
\item[Modellierung] \hfill \\
Mittels des Editors wird ein Modell angefertigt und abgespeichert. In dem Modell sind alle Arten von Instruktionen enthalten, welche auch alle mindestens einmal über einen Pfad des Konversationsroutings ausgeführt werden. Das Routing muss mindestens einen Zyklus in seinem Ablauf vorweisen. Zusätzlich sind mindestens je eine benutzerdefinierte Variable und Funktion im Routing enthalten, die auch in einer oder mehreren Script-Instruktionen verwendet werden. Bei der Modellierung des Routings müssen alle Editoren (Script-, Listen-, Ausdrucks- und Routing-Editor) mindestens einmal ausgeführt werden. Nach dem Editieren eines jeden Wertes muss durch erneutes Editieren überprüft werden, ob der Wert auch tatsächlich übernommen wurde. Treten während des Modellierens keine Fehler auf, wird das Routing im Dateisystem abgespeichert. 
\item[Ausführung] \hfill \\
Als nächstes wird die Routing Engine gestartet. Diese ist so konfiguriert, dass sie beim Start das im vorherigen Schritt modellierte Routing lädt und bei einem eingehenden Anruf ausführt.
\item[Anruf] \hfill \\
Nachdem die Routing Engine gestartet ist, kann das Modell über Anrufe getestet werden. Es werden so viele Anrufe getätigt, dass jeder Pfad des Routings einmal ausgeführt wurde und sich das Verhalten mit der Spezifikation des Modells deckt. Insbesondere wird stichprobenartig für jeden Pfad getestet, wie sich das System verhält wenn entweder der Anrufer oder ein beteiligter Agent frühzeitig den Anruf beendet. In diesen Fällen dürfen für ein Bestehen des Tests keine Anrufe auf Agenten- oder Kundenseite unbeantwortet übrig bleiben.
\end{description}
Ein Kandidat für ein Test-Routing, mit dem die oben stehenden Schritte ausgeführt werden, ist in Abbildung \ref{fig:TestRouting} zu sehen. Neben dem Kriterium, alle Instruktionen zu beinhalten, bietet es eine überschaubare Anzahl an Pfaden die durch das Routing  führen und kann so mit wenigen Anrufen ganz abgedeckt werden. Auf der Abbildung sind die Skripte sowie die benutzerdefinierten Variablen und Funktionen, die zum Einsatz kommen, nicht zu sehen. Angelegt ist die Integer-Variable \texttt{Count}, welche in der Set Variables-Instruktion mit null (in Ziffern: 0) initialisiert wird. Anschließend wird in der Branch-Instruktion abgefragt, ob \texttt{Count} größer als zehn ist und ob es sich um eine gerade Zahl handelt. Letzteres wird mit einer benutzerdefinierten Funktion \texttt{IsEven} überprüft. Die Branch-Instruktion nimmt also zuerst den False-Ausgang, welcher in das erste Skript führt. Dort wird die \texttt{Count}-Variable um eins inkrementiert und führt zurück in die Branch-Instruktion. Diese Schleife wird solange ausgeführt, bis die Branch-Instruktion den True-Ausgang nimmt. Dieser führt zum zweiten Skript, in der die benutzerdefinierte Funktion \texttt{PrintTime} aufgerufen wird, welche die aktuelle Systemzeit auf der Konsole ausgibt.

\begin{figure} %[hbtp]
	\centering
		\includegraphics[width=\textwidth]{img/TestRouting.png}
	\caption[DSL-Modell für manuelle Tests]{Zu sehen ist das DSL-Modell, welches im Zuge von automatisierten und manuellen  Tests zum Einsatz kommt.}
	\label{fig:TestRouting}
\end{figure}

\section{Metriken}
Bei der Umsetzung der DSL entstehen zwei Arten von Code. Die eine Art von Quellcode setzt das System um: Es handelt sich hierbei um den Quellcode, der den Editor, den Transformator und die Modell-Validierung implementiert. Die andere Art von Code ist der vom System generierte Code, also der CIL-Bytecode, der ein Routing umsetzt. Die erste Code-Art ist für den weiteren Lebenszyklus der DSL von unmittelbarer Bedeutung und muss daher vor allem wartbar sein. Daher folgt eine Einschätzung zum vermessenen Umfang und Komplexität dieses Code-Anteils. Diese Messungen für den generierten Byte-Code durchzuführen, wäre  wenig aussagekräftig, da sowohl Umfang als auch Komplexität stark vom transformierten Modell abhängt: Je komplexer und umfangreicher das Konversationsrouting, desto komplexer ist der Code, der dieses Konversationsrouting umsetzt. Die Grenzen sind vor allem nach oben nicht abschätzbar, da der Benutzer in Script-Instruktionen und eigenen Funktionen ``unvorhersehbaren'' Code einfügen kann. Da außerdem aus der erstellten Syntax für Benutzer unleserlicher Bytecode generiert wird, welcher nicht dafür gedacht ist per Hand gewartet zu werden, ist das Berechnen von detaillierten Metriken für den so erstellten Code wenig zielführend. 

\subsection{Geschriebener Code}
Berechnet wurden die folgenden Metriken mit der Entwicklungsumgebung Visual Studio 2017 Ultimate. Zum Einsatz kommen die ``Executable Lines of Code'', bei der die ausführbaren Bytcode-Zeilen gemessen werden, um einen generellen Überblick über den Umfang der Code-Basis zu liefern, und die zyklomatische Komplexität, auch als McCabe-Metrik bekannt, welche den Grad an Verzweigungen im Programmfluss beschreibt. Beide Metriken sind im Feld der Software-Architektur verbreitete Maße, deren Ermittlung jeweils bei \cite[S. 35ff]{Laird:06} und \cite[S. 58ff]{Laird:06} nachgelesen werden kann. Da die Angaben für gesamte Namespaces gemacht werden, wird bei der zyklomatischen Komplexität der Mittelwert und der Median über alle Methoden dieses Namespaces angegeben. Alle der genannten Namespaces sind Unter-Namespaces von \texttt{ilogixx.ConversationFlow}. In Tabelle \ref{tab:metrikenGeschriebenerCode} sind die gemessenen Metriken für den im Zuge der vorliegenden Arbeit händisch verfassten Code zu sehen.
\begin{center}
    \begin{tabular}{| p{0.18\textwidth} | p{0.18\textwidth} | p{0.18\textwidth} | p{0.18\textwidth} | p{0.18\textwidth} |}
    \hline
    Komponente & Namespace & LOC & zykl. Kompl. (mittel) & zykl Kompl. (median)\\ \hline
    Objekt-Graph & Core & 183 & 1,146341463 & 1 \\ \hline
    Editor & UI & 2818 & 2,212669683 & 1 \\ \hline
    Transformator & Generation & 742 & 1,715846995 & 1 \\ \hline
    Validator & Validation & 186 & 1,806451613 & 1 \\ \hline
    \end{tabular}
    \label{tab:metrikenGeschriebenerCode}
\end{center}
Die Klassenstruktur des Objekt-Graphen weist mit der geringsten Anzahl an Lines of Code und der kleinsten durchschnittlichen zyklomatischen Komplexität die niedrigste Komplexität auf. Dies ist nicht überraschend, da es sich um eine fast reine Datenstruktur handelt, die wenig programmatisches Verhalten aufweist. An zweiter Stelle kommt die Komponente der Modell-Validierung. Komplexe Abläufe und ein hoher Umfang an Code konnte hier vermieden werden, da die meiste Arbeit, welche die Metriken erhöhen könnte, an andere Komponenten ausgelagert wird, wie zum Beispiel den Transformator. Dieser weist eine höhere Verzweigung des Programmablaufs und einen größeren Umfang auf, wie die Metriken zum Ausdruck bringen. Dennoch bleibt auch hier die durchschnittliche McCabe-Metrik deutlich unter dem empfohlenen Wert von zehn, auch dank der rekursiven Vorgehensweise dieser Komponente. Durch diese werden Schleifen vermieden, welche einen hohen Einfluss auf die Anzahl an mögliche Programmpfaden haben. Der hohe Anteil an Lines of Code für den Transformator lässt sich durch den Gebrauch der Roslyn-API begründen, denn im Zuge des Aufbaus eines kompletten CIL-Syntaxbaums werden im Vergleich zu anderen Komponenten  viele API-Aufrufe benötigt. 
\newline
Die laut den oben stehenden Metriken komplexeste Komponente der DSL ist der Editor. Der im Vergleich extrem hohe LOC-Wert für diese Komponente lässt sich damit begründen, dass automatisch generierter Code des Devexpress-Framework in diesem Namespace enthalten ist. Devexpress benutzt einen sogenannten Component Designer, welcher ``Boilerplate''-Code zum Erstellen von Windows Forms-Komponenten automatisch generiert. Dieser Code ist sehr umfangreich, fällt aber wenig ins Gewicht, da der vom Component Designer generierte Code nicht von Hand lesbar oder editierbar sein soll und damit nicht die Komplexität des System widerspiegelt. Der hohe Mittelwert der zyklomatischen Komplexität ist auf die Fallunterscheidungen zurückzuführen, die auf View- und ViewModel-Ebene durchgeführt werden, wenn es um die Behandlung von Events geht, welche durch die Benutzung der grafischen Benutzeroberfläche ausgelöst werden.
\newline
Erwähnenswert ist weiterhin, dass keine Komponente beim Median der zyklomatischen Komplexität einen Wert von eins überschreitet. In Verbindung mit dem geringen Mittelwert aller Komponenten spricht dies für eine hohe Geradlinigkeit des Programmflusses. 

\section{Performanz}
Sowohl der generierte als auch der händisch verfasste Code muss eine hohe Leistungsfähigkeit aufweisen. Aber ähnlich wie im vorangegangen Kapitel gestaltet sich die Abschätzung der Performanz für den generierten Bytecode als schwierig: Der Code kann nur so effizient sein, wie der Benutzer ihn gestaltet. Zusätzlich haben Maße wie Zeitmessungen wenig Bedeutung, da ein Konversationsrouting an vielen Stellen in erwünschten Maßen blockiert. Eine Laufzeitanalyse ist in diesem Fall ebenfalls wenig zielführend, da die Laufzeit des Algorithmus, der ein Konversationsrouting ausführt, von keiner Eingabe abhängig ist, sondern höchstens von der Interaktion mit einem Anrufer.
\newline
Aus diesem Grund konzentriert sich die folgende Analyse auf den Code, welcher die Transformation durchführt. Auch hier ist eine hohe Performanz wichtig, da dieser Code im Zuge der Modell-Validierung und damit laufend während der Modellierung ausgeführt wird. Leistungsfähiger Code kann hier also dafür sorgen, dass der Benutzer schnelles Feedback über etwaige Fehler bekommt und bei der Modellierung nicht lange auf die Transformation warten muss.

\subsection{Geschriebener Code}

\subsubsection{Abschätzung der Komplexität}
\label{subsubsec:Abschaetzung der Komplexitaet}
Bei der Transformation eines DSL-Modells wird ein solches wie ein Graph aufgefasst und entsprechend einer Depth-First-Vorgehensweise traversiert. Sei ein Graph definiert als $G = (V, E)$ mit $V$ als der Menge an Knoten und $E \subseteq V^{2}$ als die Menge an Kanten. Laut \cite[S. 479]{Cormen:90} befindet sich die Komplexität eines Depth-First-Algortihmus dann in der Komplexitätsklasse $O(\left\vert{V}\right\vert + \left\vert{E}\right\vert)$. Bei der Interpretation des Objekt-Graphen als klassischen Graph werden alle \texttt{Input}-, \texttt{Node}- und \texttt{Output}-Instanzen als Knoten, und alle Referenzen zwischen diesen Instanzen als die Kanten eines Graphen aufgefasst. Während eine Traversierung des Objekt-Graphen also mit linearer Zeitkomplexität möglich ist, gestaltet sich eine genaue Bestimmung der Komplexitätsklasse des Transformationsalgorithmus als schwierig. Dies liegt vor allem an der Benutzung der Roslyn-API, deren Methoden eine unbekannte Zeitkomplexität haben. Vor allen die API-Aufrufe zur Erstellung der instruktionsspezifischen Syntax, die für jede besuchte Instruktion im Routing durchgeführt werden muss, erhöhen die Laufzeit.
\newline
Aufgrund dieser unvollständigen Informationslage wird an dieser Stelle von einer formalen Laufzeitanalyse abgesehen. Stattdessen werden empirische Zeitmessungen präsentiert.

\subsubsection{Zeitmessungen}
Der folgende Versuch soll eine ungefähre Vorstellung davon geben, wie lange die Transformationskomponente für die Generierung einer Assembly benötigt und in welchem Verhältnis diese Dauer zum Umfang eines zu transformierenden DSL-Modells steht. Dafür wurden DSL-Modelle mit einer Anzahl an Instruktionen zwischen eins und hundert transformiert, während jeweils die Dauer der Syntax-Generierung und der anschließenden Kompilierung durch die Roslyn-API gemessen wurde. Zwischen je zwei Transformations-Versuchen wurde die Anzahl der Instruktionen um eine einzelne MediaPlayback-Instruktion erhöht, sodass eine eindimensionale wachsende Kette an Instruktionen zu einer Assembly konvertiert wurde. Der Aufbau dieser Konversationsroutings ist zwar nicht repräsentativ für den realen Anwendungsfall, ermöglicht aber den Vergleich zwischen Testergebnissen von Konversationsroutings mit sehr vielen Instruktionen. Die Schritte des Ladens der Protobuf-Datei und der Deserialisierung wurden dabei ausgelassen, da diese in dem Anwendungsszenario der kontinuierlichen Transformation während der Modellierung durch den Benutzer lediglich einmal und nicht wiederholt ausgeführt werden. Für jedes DSL-Modell wurde die durchschnittliche Dauer aus hundert konsekutiven Transformationen gemessen, um das Testergebnis vor zufälligen Abweichungen der Laufzeitumgebung und des Betriebssystems abzuschirmen. Zusätzlich ist die Initialisierungszeit, die die Roslyn-API beim ersten Gebrauch benötigt, nicht in den Messungen enthalten. Alle Messungen wurden auf einem Intel Core i7-4770 mit 3,40 GHz durchgeführt. Die Ergebnisse sind in Abb. \ref{fig:AverageTimeDiagram} zu sehen. 
\newline
Zu beobachten ist unter anderem ein annähernd linearer Anstieg der Kompilierungsdauer in Abhängigkeit zur Instruktionsanzahl. Bei der Dauer der Syntaxgenerierung hingegen zeichnet sich ein quadratischer Wuchs ab. Dennoch nimmt auch bei einer Anzahl von hundert Instruktionen die Kompilierung der Syntax den größten zeitlichen Aufwand in Anspruch, während die Syntax-Generierung mit wenigen Millisekunden immer noch vergleichsweise kurz ist. Zu beachten ist hierbei auch, dass Eingabemengen von hundert Instruktionen über die Anforderungen des normalen Anwendungsfalls hinaus gehen.

\begin{figure} %[hbtp]
	\centering
		\includegraphics[width=0.84\textwidth]{img/AverageTimeDiagram.png}
	\caption[Durchschnittliche Dauer von Assembly-Generierungen]{Zu sehen ist die durchschnittliche Dauer einer Assembly-Generierung in Abhängigkeit der Instruktions-Anzahl eines Konversationsroutings. Der Durchschnitt für eine bestimmte Anzahl an Instruktionen wurde jeweils aus hundert konsekutiven Assembly-Generierungen errechnet.}
	\label{fig:AverageTimeDiagram}
\end{figure}
\chapter{Zusammenfassung und Ausblick}
TODO
% ...
%--------------------------------------------------------------------------
\backmatter                        		% Anhang
%-------------------------------------------------------------------------
\bibliographystyle{geralpha}			% Literaturverzeichnis
\bibliography{literatur}     			% BibTeX-File literatur.bib
%--------------------------------------------------------------------------
\printindex 							% Index (optional)
%--------------------------------------------------------------------------
%\begin{appendix}						% Anhänge sind i.d.R. optional
   \chapter{Anhang}

\section{Beigefügter Quellcode}
Der vorliegenden Arbeit ist der im Zuge des Projektes durch den Autor entwickelte Quellcode beigefügt. Für jeden im Code enthaltenen Namespace ist in der Beilage ein Ordner angelegt, in dem die Dateien mit dem Quellcode dieses Namespaces zu finden sind. Dateinamen, die mit ``.Designer.cs''  enden, enthalten automatisch generierten Code des Windows Forms-Framework und sind der Vollständigkeit halber beigefügt. Die Namespaces und ihr Bezug zu den Komponenten der implementierten DSL ist in der untenstehenden Tabelle aufgelistet. Alle aufgeführten Namespaces sind Unter-Namespaces von \texttt{ilogixx.ConversationFlow}.

\begin{table}[hbtp]
\centering
\settowidth\tymin{\textbf{ilogixx.ConversationFlow.PropertyEditors}}
%\setlength\extrarowheight{2pt}
\begin{tabulary}{1.0\textwidth}{L|L}
\textbf{Namespace} & \textbf{Inhalt} \\
\hline

Core & Enthält die Klassendefinitionen für den Objekt-Graphen \\
\hline

Generation & Enthält Quellcode des Transformators \\
\hline

Generation.Tests & Enthält Modultests für den Transformator\\
\hline

PropertyEditors & Enthält Hilfsklassen, die verwendet werden, um grafische Benutzeroberflächen zur Spezifizierung von Instruktionsparametern anzuzeigen \\
\hline

TestTools & Enthält Hilfsklassen für die Modultests\\
\hline

UI & Enthält den Quellcode des Editors \\
\hline

UI.Tests & Enthält Modultests für den Editor \\
\hline

Validation & Enthält den Quellcode, der die Modell-Validierung umsetzt \\
\hline

Validation.Tests & Enthält Modultests für die Modell-Validierung \\

\end{tabulary}
\caption{\textit{Die Namespaces des beigefügten Codes und was sie beinhalten.}}
\label{tab:namespaces}
\end{table}			% Glossar   
   \include{chapters/Selbststaendigkeitserklaerung}	% Selbstständigkeitserklärung
%\end{appendix}

\end{document}
