\chapter{Anhang}

\section{Beigefügter Quellcode}
Der vorliegenden Arbeit ist der im Zuge des Projektes durch den Autor entwickelte Quellcode beigefügt. Für jeden im Code enthaltenen Namespace ist in der Beilage ein Ordner angelegt, in dem die Dateien mit dem Quellcode dieses Namespaces zu finden sind. Dateinamen, die mit ``.Designer.cs''  enden, enthalten automatisch generierten Code des Windows Forms-Framework und sind der Vollständigkeit halber beigefügt. Die Namespaces und ihr Bezug zu den Komponenten der implementierten DSL ist in der untenstehenden Tabelle aufgelistet. Alle aufgeführten Namespaces sind Unter-Namespaces von \texttt{ilogixx.ConversationFlow}.

\begin{table}[hbtp]
\centering
\settowidth\tymin{\textbf{ilogixx.ConversationFlow.PropertyEditors}}
%\setlength\extrarowheight{2pt}
\begin{tabulary}{1.0\textwidth}{L|L}
\textbf{Namespace} & \textbf{Inhalt} \\
\hline

Core & Enthält die Klassendefinitionen für den Objekt-Graphen \\
\hline

Generation & Enthält Quellcode des Transformators \\
\hline

Generation.Tests & Enthält Modultests für den Transformator\\
\hline

PropertyEditors & Enthält Hilfsklassen, die verwendet werden, um grafische Benutzeroberflächen zur Spezifizierung von Instruktionsparametern anzuzeigen \\
\hline

TestTools & Enthält Hilfsklassen für die Modultests\\
\hline

UI & Enthält den Quellcode des Editors \\
\hline

UI.Tests & Enthält Modultests für den Editor \\
\hline

Validation & Enthält den Quellcode, der die Modell-Validierung umsetzt \\
\hline

Validation.Tests & Enthält Modultests für die Modell-Validierung \\

\end{tabulary}
\caption{\textit{Die Namespaces des beigefügten Codes und was sie beinhalten.}}
\label{tab:namespaces}
\end{table}





\section{Kurzreferenz}
\label{sec:Kurzreferenz}
Tabelle \ref{tab:kurzreferenz} fasst die wichtigsten Begriffe der vorliegenden Ausarbeitung zusammen.

\begin{table}[hbtp]
\centering
\settowidth\tymin{\textbf{Erklärung}}
%\setlength\extrarowheight{2pt}
\begin{tabulary}{1.0\textwidth}{L|L}
\textbf{Begriff} & \textbf{Erklärung} \\
\hline
Abstrakte Syntax & Repräsentation einer Programmsyntax ohne Informationen über die Darstellung der Syntax.\\

\hline

ACD & Abk. für automatic conversation distribution. Beschreibt die automatische Verteilung von Konversation an qualifizierte Agenten. Siehe Abs. \ref{subsec:Automatische Kontaktverteilung} \\

\hline

Agent & Teilnehmer eines Contactcenters, der Konversationen führt. Ein Agent ist für eine oder mehrere Sprachen und Wissensbereiche qualifiziert. Siehe Abs. \ref{subsec:Agenten}. \\

\hline

CIL & Abk. für Common Intermediate Language. Die Sprache, in die Code von Benutzern des .NET-Frameworks übersetzt wird. \\

\hline

DSL & Abk. für domain-specific language, Fachbegriff für eine domänenspezifische Sprache zur Modellierung von Programmabläufen. Wird benutzt, um die im Zuge der vorliegenden Arbeit implementierte Sprache zu referenzieren.\\

\hline

DSL-Modell & Ein in einer DSL spezifizierter Programmablauf.\\

\hline

Instruktion & Bestandteil eines DSL-Modells in der implementierten DSL. Eine Instruktion führt eine bestimmte Operation auf einer eingehenden Konversation aus. Instruktionen werden auf Objekt-Graph auf \texttt{Node}-Subtyp-Instanzen abgebildet. \\

\hline

Konkrete Syntax & Visuelle Darstellung der Programmsyntax. Endbenutzer modellieren (im Fall einer DSL) oder programmieren (im Fall einer Universalsprache), indem sie die konkrete Syntax gestalten.\\

\hline

Konversation & Ein Informationsaustausch zwischen zwei Teilnehmern über einen beliebigen Kommunikationskanal (z.B. Telefon, Email, Web Chat etc.)\\

\hline

Konversationsrouting & Ein in der DSL spezifizierter Ablauf zu Bearbeitung einer eingehenden Konversation durch die Routing Engine. Jedes Konversationsrouting ist ein Modell der implementierten DSL.\\

\hline

MyCC & Abk. für MyContactCenter, einer Software zum Betreiben eines Contactcenters. Siehe Abs. \ref{sec:MyContactCenter}.\\

\hline

.NET & Von Microsoft entwickeltes Framework zur Entwicklung von Software. Siehe \ref{subsec:DotNet}.\\

\hline

Objekt-Graph & Datenstruktur von Objekt-Instanzen die durch gegenseitiges referenzieren einen Graph bilden. Wird in der implementierten DSL zum Speichern der abstrakten Syntax benutzt.\\

\hline

PBX & Abk. für Private Branch Exchange, der Software welche eine Telefonanlage implementiert. Siehe \ref{subsec:IP-Private Branch Exchange}.\\

\hline

Roslyn & .NET-Compiler-Plattform, welche es erlaubt CIL-Syntax zu generieren und zu kompilieren. Siehe \ref{subsec:Roslyn}.\\

\hline

Routing Engine & MyCC-Komponente, die Konversationsroutings transformiert und den entstehenden CIL-Bytecode für eingehende Konversationen ausführt. Siehe \ref{subsec:Routing Engine}.\\

\hline
SIP & Abk. für Session Initiation Protocol, ein Protokoll zum Aushandeln von Kommunikationssitzungen. Wird für die Kommunikation zwischen der Routing Engine und der PBX benutzt. \ref{subsec:Session Initiation Protocol}.  \\

\end{tabulary}
\caption{\textit{Eine Kurzreferenz der verwendeten Begriffe}}
\label{tab:kurzreferenz}
\end{table}