%Praesentationsmodus
\documentclass[t,aspectratio=169,divpsnames]{beamer}
%Die Beameroption aspectratio legt das verwendete Seitenverhaeltnis fest
%aspectratio=169	16:9 Seitenverhaeltnis
%aspectratio=1610	16:10 Seitenverhaeltnis
%aspectratio=43		4:3 Seitenverhaeltnis
%Die Beameroption envcountsect nummeriert Umgebungen wie theorem pro section durch.
%Die Beameroption divpsnames wird an das xcolor Paket durchgereicht.

%Handout-Generierung mit Foliennotizen (statt obiger Zeile für den Präsentationsmodus verwenden)
%\documentclass[t,handout,aspectratio=169]{beamer}
%\setbeameroption{show notes}

\usepackage[utf8]{inputenc}

% Deutsch
\usepackage[ngerman]{babel} 
\usepackage{bibgerm}

% Englisch
%\usepackage[english]{babel}

\input{templatesetup}

% Stil des Literaturverzeichnisses
%\bibliographystyle{geralpha}
%\bibliographystyle{alpha}
\bibliographystyle{abstract}

%Bitte ausfuellen:
\title[Entwurf und Implementierung einer DSL zur Spezifikation von Konversationsrouting]{Entwurf und Implementierung einer grafischen, domänenspezifischen Sprache zur Spezifizierung des Konversationsrouting in Contactcentern}
%\subtitle{Titelzusatz}
\author{David Wichter}
\institute{Hochschule Trier}
\date{28.02.2018}
\subject{Thema für PDF-Metadaten (optional)}

%Inhaltsverzeichnisses bis auf subsubsection-Ebene:
%\setcounter{tocdepth}{3}

%Aktivieren, um am Anfang jeder Section ein Inhaltsverzeichnis zur Section anzuzeigen
%\AtBeginSection[]
%{
%\begin{frame}<beamer>
%\frametitle{Agenda}
%\tableofcontents[currentsection,hideothersubsections,sectionstyle=show/hide,subsubsectionstyle=show/show]
%\end{frame}
%}

%Aktivieren, um alles Schritt-fuer-Schritt einzublenden
\beamerdefaultoverlayspecification{<+->}
\setbeamercovered{invisible}

\begin{document}

\begin{frame}
\titlepage
\end{frame}

\begin{frame}
	\frametitle{Agenda}
	\tableofcontents
	%\tableofcontents[hideallsubsections] % Subsections ausblenden
	%\tableofcontents[pausesections] %Sections Schritt-fuer-Schritt einblenden
\end{frame}

\begin{frame}{Einführung}{Wer?}
	\only<1->
		{
			\begin{figure}
				\includegraphics[scale=0.08]{img/ilogixx_logo.png}
			\end{figure}
		}
	\only<2->
		{
			\begin{figure}
				\includegraphics[scale=0.65]{img/mycontactcenter_logo.jpeg}
			\end{figure}
		}
\end{frame}

\begin{frame}{Einführung}{Was?}
	
\end{frame}

\begin{frame}{Einführung}{Wie?}

\end{frame}

\section*{Schluss}
\begin{frame}
	\begin{center}
		\huge{Vielen Dank für die Aufmerksamkeit}
	\end{center}
	\begin{center}
		\Huge{Fragen?}
	\end{center}
\end{frame}

\begin{frame}[allowframebreaks]{\bibname}
\bibliography{literatur}     %BibTeX-Datei literatur.bib
\end{frame}


\end{document}
