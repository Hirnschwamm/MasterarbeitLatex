%Praesentationsmodus
\documentclass[t,aspectratio=169,divpsnames]{beamer}
%Die Beameroption aspectratio legt das verwendete Seitenverhaeltnis fest
%aspectratio=169	16:9 Seitenverhaeltnis
%aspectratio=1610	16:10 Seitenverhaeltnis
%aspectratio=43		4:3 Seitenverhaeltnis
%Die Beameroption envcountsect nummeriert Umgebungen wie theorem pro section durch.
%Die Beameroption divpsnames wird an das xcolor Paket durchgereicht.

%Handout-Generierung mit Foliennotizen (statt obiger Zeile für den Präsentationsmodus verwenden)
%\documentclass[t,handout,aspectratio=169]{beamer}
%\setbeameroption{show notes}

\usepackage[utf8]{inputenc}

% Deutsch
\usepackage[ngerman]{babel} 
\usepackage{bibgerm}

% Englisch
%\usepackage[english]{babel}

\usepackage{calc}  
%\usepackage{enumitem}  

\input{templatesetup}

% Stil des Literaturverzeichnisses
%\bibliographystyle{geralpha}
%\bibliographystyle{alpha}
\bibliographystyle{abstract}

%Bitte ausfuellen:
\title[Entwurf und Implementierung einer DSL zur Spezifikation von Konversationsrouting]{Entwurf und Implementierung einer grafischen, domänenspezifischen Sprache zur Spezifizierung des Konversationsrouting in Contactcentern}
%\subtitle{Titelzusatz}
\author{David Wichter}
\institute{Hochschule Trier}
\date{28.02.2018}
\subject{Thema für PDF-Metadaten (optional)}

%Inhaltsverzeichnisses bis auf subsubsection-Ebene:
%\setcounter{tocdepth}{3}

%Aktivieren, um am Anfang jeder Section ein Inhaltsverzeichnis zur Section anzuzeigen
%\AtBeginSection[]
%{
%\begin{frame}<beamer>
%\frametitle{Agenda}
%\tableofcontents[currentsection,hideothersubsections,sectionstyle=show/hide,subsubsectionstyle=show/show]
%\end{frame}
%}

%Aktivieren, um alles Schritt-fuer-Schritt einzublenden
%\beamerdefaultoverlayspecification{<+->}
\setbeamercovered{invisible}

\defbeamertemplate{description item}{align left}{\insertdescriptionitem\hfill}

\begin{document}

\begin{frame}
\titlepage
\end{frame}

%\begin{frame}
%	\frametitle{Agenda}
%	\tableofcontents
%	%\tableofcontents[hideallsubsections] % Subsections ausblenden
%	%\tableofcontents[pausesections] %Sections Schritt-fuer-Schritt einblenden
%\end{frame}

\begin{frame}{Einführung}{Wer?}
	\only<1->
	{
		\begin{figure}
			\includegraphics[scale=0.08]{img/ilogixx_logo.png}
		\end{figure}
	}
	\only<2->
	{
		\begin{figure}
			\includegraphics[scale=0.65]{img/mycontactcenter_logo.jpeg}
		\end{figure}
	}
\end{frame}

\begin{frame}{Einführung}{Was?}
	\begin{columns}[T]
		\setbeamertemplate{description item}[align left]
		\begin{column}{0.55\textwidth}
			\only<2->
			{
				\begin{description}[MyCC-Server]
					\item<2->[Konversation] Informationsaustausch über beliebigen Kanal
					\item<3->[ACD] Automatic Conversation Distribution	
					\item<4->[Agent] Contactcenter-Teilnehmer
					\item<5->[MyCC-Server] Zentrale Verwaltungskomponente		
					\item<6->[IP-PBX] IP-Private Branch Exchange
					\item<7->[SIP] Session Initiation Protocol
				\end{description}
			}
		\end{column}
		\begin{column}{0.45\textwidth}
				\only<1-7>
				{
					\vspace{0.5cm}
        			\includegraphics[scale=0.4]{img/MyCC_logo.jpg}
				}        		
        		\only<8>
				{
					\vspace{0.5cm}
					\includegraphics[scale=0.15]{img/MyCCStructure.png}
				}
    	\end{column}
    \end{columns}
\end{frame}

\begin{frame}{Einführung}{Was?}
	\center
	\includegraphics[width=0.9\textwidth]{img/RoutingEngineSipExplanation.png}
\end{frame}

\begin{frame}{Einführung}{Was?}
	\begin{columns}
		\begin{column}{0.5\textwidth}
			\begin{itemize}
				\item Charakteristiken domänenspezifischer Sprachen (nach \cite[S. 30ff]{Voelter:13})
				\begin{itemize}
					\item<2-> Domäne vs. General Purpose
					\item<3-> Modellieren vs. Programmieren
					\item<4-> Abstraktion vs. Maschinennähe
				\end{itemize}
			\end{itemize}
		\end{column}
		\begin{column}{0.5\textwidth}
			\only<2>
			{
				\center
				\includegraphics[width=0.4\textwidth]{img/Druckpresse.jpg}
				\includegraphics[width=0.58\textwidth]{img/TuringMachine.jpg}
			}
		\end{column}
	\end{columns}
\end{frame}

\begin{frame}{Einführung}{Warum?}
	\begin{itemize}
		\item<1-> Kunden von MyCC wollen...
			\begin{itemize}
				\item<2-> Anpassbarkeit 
				\item<3-> Benutzerfreundlichkeit
				\item<4-> Funktionsumfang
			\end{itemize}
		\item<5-> Gewinne für MyCC sind...
			\begin{itemize}
				\item<6-> Flexibilität
				\item<7-> Unabhängigkeit
				\item<8-> Skalierbarkeit
			\end{itemize}
	\end{itemize}
\end{frame}

\section*{Schluss}
\begin{frame}
	\begin{center}
		\huge{Vielen Dank für die Aufmerksamkeit}
	\end{center}
	\begin{center}
		\Huge{Fragen?}
	\end{center}
\end{frame}

\begin{frame}[allowframebreaks]{\bibname}
\bibliography{literatur}     %BibTeX-Datei literatur.bib
\end{frame}


\end{document}
