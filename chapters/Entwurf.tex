\chapter{Entwurf}

\section{Anforderungen}
Einige Teilmenge der Anforderungen der domänenspezifischen Sprache wurden bereits in Kapitel \ref{sec:Motivation} in Form von User-Stories aufgeführt. Im Folgenden werden die Anforderungen in einer allgemeineren und umfassenderen Form wiedergegeben. Die zentrale Aussage die den Anforderungen zu Grunde liegt ist die zusammenfassende Formulierung des zu lösenden Problems: ``Benutzer von MyContactCenter brauchen eine Möglichkeit, das Routing von Konversationsanfragen für eine automatische Kontaktverteilung frei programmieren zu können''. Die Umsetzung dieser Vision ist das langfristige Ziel, sprengt allerdings den Rahmen einer Masterarbeit und damit den der vorliegenden Ausarbeitung. Zwar wurde beim Entwurf Acht darauf gegeben, allen Arten von Konversationsanfragen gerecht zu werden. Für die vorliegende Implementierung beschränkt sich die Ausarbeitung aber lediglich auf das Routing von eingehenden Telefonanrufen. Die relevante Aufgabe lautet also: ``Benutzer von MyContactCenter brauchen eine Möglichkeit, das Routing eingehender Telefonanrufe für eine automatische Kontaktverteilung frei programmieren zu können''.
\newline
Aus dieser Aufgabe ergeben sich folgende Anforderungen für die DSL:
\begin{itemize}
\item Die DSL muss eine Sammlung an Instruktionen bereit stellen, welche mit einem eingehenden Anruf interagieren
\item Instruktionen der DSL müssen sich in einer Reihenfolge arrangieren lassen, welche den zeitlichen Ablauf eines Anrufroutings spezifiziert. Ein solcher Ablauf muss Verzweigungen und Schleifen zulassen.
\item Die Sammlung an Instruktionen müssen folgende Interaktionen mit einem eingehenden Anruf möglich machen:
	\begin{itemize}
	\item Entgegen nehmen eines Anrufs
	\item Abspielen von Audiodateien 
	\item Abrufen von Eingaben des Anrufers über Mehrfrequenzwahlverfahren
	\item Ausführen von Benutzerscripten. Bei Bedarf soll ein Benutzer über die Programmiersprache C\# eigene Scripte programmieren können, welche dann im Routing ausgeführt werden.
	\item Einordnen von Anrufen in die Kategorien ``Sprache'' und ``Wissensbereich''. Die Einordnung eines Anrufes in diese Kategorien bestimmt, welche Agenten den Anruf entgegen nehmen können.
	\item Terminieren eines Anrufs
	\item Zustellung eines Anrufs an einen Agenten
	\end{itemize}
\item Ein Ablauf von Instruktionen muss mit einer grafischen Benutzeroberfläche arrangierbar sein.
\item Ein Ablauf von Instruktionen muss in ausführbaren Code transformiert werden können.
\item Der Ersteller eines Ablaufs von Instruktionen soll durch eine hohe Benutzerfreundlichkeit unterstützt und davor bewahrt werden, unnötige Fehler zu machen.  
\end{itemize}
Die obenstehende Liste stellt die Mindestanforderungen dar, die erfüllt sein müssen, damit ein Benutzer über die gewünschten Konfigurationsmöglichkeiten für eine Automatische Kontaktverteilung verfügt. Der im Folgenden Kapitel beschriebene Entwurf einer domänenspezifischen Sprache ist ein Versuch, ein System zu designen, welches die obenstehenden Anforderungen erfüllt. Zur Umsetzung des Entwurfs wurde ein grafischer Editor und ein Code-Generator umgesetzt, deren Implementierung näher in Kapitel \ref{chap:Implementierung} erläutert wird.

\section{Beschreibung der Sprache}
Die domänenspezifische Sprache ist als grafische Sprache entworfen. Das bedeutet, dass die konkrete Syntax, also das äußere Erscheinungsbild, mit dem der Benutzer interagiert, in Formen und Verbindungen statt in Text ausgedrückt wird. Vergleichbar sind grafische DSLs zum Beispiel mit der Unified Modeling Language (UML), welche allerdings den Anspruch erhebt, eine Universalsprache zu sein [CITATION].
\newline 
In der vorliegend implementierten DSL werden Formen und Linien im zweidimensionalen Raum angeordnet, um das Verhalten des Konversationsroutings zu spezifizieren. Instruktionen, die mit einer eingehenden Kontaktanfrage interagieren, sind dabei als Rechtecke dargestellt. Es gibt eine Vielzahl von verschiedenen Instruktionen, welche einzeln in folgenden Abschnitten beschrieben werden. Jede Art von  Instruktion ist mit einem Namen eindeutig gekennzeichnet und kann einmal, mehrmals, oder, mit Ausnahme der Start-Instruktion, auch gar nicht im Routing auftreten. Das Verhalten von Instruktionen kann parametrisiert sein: Zum Beispiel kann ein Benutzer bei der Instruktion zum Abspielen einer Audio-Datei die Datei angeben, die wiedergegeben werden soll. Jedes Rechteck besitzt eine bestimmte Anzahl von Ein- und Ausgängen, symbolisiert durch kleinere Rechtecke, die in der Peripherie des großen Rechtecks eingelassen ist. Die Anzahl an Ein- und Ausgängen eines Rechtecks bestimmen, wie viele ein- beziehungsweise ausgehende Verbindungen für ein Rechteck zugelassen werden und sind für jede Art von Instruktion individuell festgelegt. Eine gerichtete Verbindung in Form eines Pfeils kann zwischen einem Ein- und Ausgang gezogen werden. Dabei geht der Pfeil vom Eingang aus und zeigt mit der Pfeilspitze auf den Eingang. Während ein Ausgang immer nur eine einzige ausgehende Verbindung hat, kann ein Eingang beliebig viele eingehende Verbindungen akzeptieren. Die Verbindungen zwischen Rechtecken symbolisieren den zeitlichen Ablauf innerhalb des Konversationsroutings: Eine Instruktion A wird dann ausgeführt, wenn eine andere Instruktion B beendet wurde und im Zuge der Ausführung von B ein Ausgang gewählt wurde, der über eine Verbindung zum Eingang von A führt. Welcher Ausgang von B gewählt wird, hängt dabei von der Instruktion ab, die von B symbolisiert wird. Jedes spezifizierte Routing benötigt einen Ausgangspunkt, an dem das Routing seinen Anfang nimmt. Dies geschieht beim Start-Knoten, welcher keinen Eingang und nur einen einzelnen Ausgang hat. Von hier aus kann nun eine Verbindung zu einer Instruktion gezogen werden, welche als erstes im Routing ausgeführt wird. 
\newline
Ein so spezifiziertes Konversationsrouting kann auch vereinfacht als gerichteter Graph betrachtet werden, in dem die Instruktionen (Rechtecke) die Knoten und die Verbindungen (Pfeile) die Kanten darstellen. Diese Betrachtungsweise vereinfacht unter anderem das Verständnis der Algorithmik welche bei der Implementierung zum Einsatz kam und ist auch der Grund, warum im Folgenden Instruktionen auch als Knoten bezeichnet werden. 

\section{Sprachelemente}
TODO

\subsection{Start}
TODO

\subsection{MediaPlayback}
TODO

\subsection{DTMF-Abfrage}
TODO

\subsection{Skills und Languages}
TODO

\subsection{Verzweigungen}
TODO

\subsection{Scripte}
TODO

\subsection{Zustellung}
TODO

\subsection{Terminierung}
TODO

\subsection{Variablen}
TODO

\subsection{Funktionen}
TODO

\section{Verarbeitungsschritte}
TODO
