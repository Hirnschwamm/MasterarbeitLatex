\chapter{Einleitung}
\label{chap:Einleitung}

DUMMY CITATION DAMIT BIBTEX SEINE SCHEISS FRESSE HÄLT UND DEN SCHEISS BAUT: \cite{Coloouris:02}

\section{MyContactCenter}
In vielen Bereichen modernen Lebens hat sich seit dem Anstoß des digitalen Zeitalters ein signifikanter Wandel eingestellt [CITATION]. Das weite Feld  der zwischenmenschlichen Kommunikation zeigt dies wie kaum ein anderes auf. Auch an der Schnittstelle zwischen Privat- und Arbeitsleben zeigt sich diese Modernisierung und im Speziellen an der Kommunikation zwischen Unternehmen und ihren Kunden. Eine moderne Institution dieser Kommunikation ist das Contact Center. Hierbei handelt es sich um einen zentralen Anlaufpunkt für Kunden, an dem diese über verschiedene Kommunikationskanäle ein Anliegen anbringen können, welches von Agenten des Unternehmens bearbeitet wird. Diese Kommunikation ist jedoch nicht einseitig: Auch Contact Center nehmen Kontakt zur Kundschaft auf, um beispielsweise Marktforschung zu betreiben. 
\newline
Moderne Call Center setzen vermehrt auf die Möglichkeiten des Internets und im speziellen auf IP-Telephonie, um Ihre Kommunikationsanforderungen zu erfüllen [CITATION]. Die von der Firma ilogixx hergestellte Software-Lösung MyContactCenter (im Folgenden mit MyCC abgekürzt) hat zum Ziel, Betreiber von Contact Center dabei zu unterstützen. Eine Kernaufgabe der Anwendung ist die Automatische Kontakt Verteilung (kurz ACD für automatic contact distribution). Das Produkt nimmt hierbei Anfragen auf unterschiedlichen Kommunikationskanälen entgegen, kategorisiert diese nach Sprache und Aufgabenbereich, und teilt sie freien Agenten zu, welche für diese Aufgabenbereiche beziehungsweise Sprache eingeteilt sind. Die Agenten interagieren nun über MyCC mit dem Anfragenden, bis die Kommunikation auf beiden Seiten beendet wird. Anschließend hat der Agent nun eine Nachbearbeitungszeit, in der das Anliegen des Anfragenden zu Ende gebracht werden kann. Danach ist der Agent wieder für die nächste Anfrage frei.
\newline
MyContactCenter ist eine verteilte Anwendung. Die Hauptkomponente ist der Server, welcher den Großteil der Verwaltungsaufgaben übernimmt. Er verwaltet eine Liste der angemeldeten Agenten, sowie deren aktuellen Kommunikationsstatus (frei, im Gespräch, etc.) und Nutzerdaten (Name, Sprache, Kompetenzbereich, etc.). Auch gehört das Kategorisieren und Zuordnen von Anfragen während der automatischen Kontakt Verteilung, und das weiterleiten dieser an den nächsten freien Agenten zu seinen Aufgaben. Konfiguriert wird der Server über einen Administrations-Client. Hier kann der Administrator des Contact Centers alle für den Betrieb benötigten Einstellungen vornehmen, wie zum Beispiel das Anlegen von neuen Sprachen und Aufgabenbereichen, oder das Verwalten von Agenten-Daten. Agenten besitzen eine eigene Client-Anwendung, die sich für die Dauer der Nutzungszeit am Server anmeldet.  Mit diesem Client kann der Agent vom Server erhaltene Anfragen bearbeiten, also zum Beispiel auf eine Email antworten oder sein IP-Telefon steuern. Dabei setzt MyCC auf bestehende Kommunikations-Infrastruktur auf. So wird beispielsweise die eigentliche IP-Telefonie nicht von MyCC implementiert, sondern von einer darunter liegenden Telefonanlage, mit der MyCC über eine Scripting-Schnittstelle kommuniziert.
\newline
Die Vermittlung einer Konversationsanfrage eines Kunden zu einem Agenten stellt also eine der Hauptaufgaben von MyContactCenter dar. Die vorliegende Ausarbeitung beschäftigt sich mit dem Entwurf und der Implementierung einer Erweiterung des Produktes, welche es den Betreibern des Contact Centers ermöglichen soll, noch mehr Kontrolle über die Behandlung der Konversationsanfrage zu erhalten. Mittels einer domänenspezifischen Sprache soll durch den Administrator das Konversationsrouting programmierbar sein, wie eine eingehende Anfrage vor und nach der Zustellung zum Agenten behandelt wird.  

\section{Motivation}
Die Motivation des vorliegenden Projektes soll zum Einstieg mit folgenden User-Stories eingeleitet werden:
\begin{itemize}
\item Als Administrator möchte ich vor dem Zustellen eines Kundenrufes eine Begrüßung abspielen.
\item Als Administrator möchte ich vor dem Zustellen eines Kundenrufes per Frequenzwahlverfahren abfragen, welche Sprache der Anrufer spricht.
\item Als Administrator möchte ich bei eingehenden Anrufen außerhalb der Ge\-schäftszeiten den Ruf ablehnen.
\item Als Administrator möchte ich bei einer eingehenden Email eine Vorlage als Standard-Antwort verschicken.
\end{itemize}
Den oben aufgelisteten Anforderungen ist nicht nur die eingenommene Kundenrolle des Administrators gemeinsam. Auch die zu erreichenden Ziele ähneln sich: Die Behandlung einer eingehender Kontaktanfrage durch die Software soll vom Kunden konfigurierbar sein. Dabei soll dem Kunden größtmögliche Freiheit geboten werden, um möglichst viele Anforderungen eines Contact Centers abzudecken. Zu den oben stehenden User-Stories können demnach noch zahlreiche weitere Beispiele mit ähnlichem Muster konstruiert werden. Die Software MyContactCenter hat den Anspruch all diesen Anforderungen gerecht werden, um in möglichst vielen Contact Centern Anwendung finden zu können. Der so resultierende Grad an individueller Konfiguration der Software durch den Anwender ist hoch. Das Problem wird bewältigt, indem es dem Benutzer möglich gemacht wird, das Verhalten von MyContactCenter im Falle einer Kontaktanfrage selber zu programmieren. Als Mittel dazu wird eine eigene grafische domänenspezifische Programmiersprache angeboten, deren Entwicklung und Implementierung Hauptgegenstand der vorliegenden Arbeit ist.
\newline
Für MyContactCenter als Software-Produkt ergibt sich bei diesem Vorgehen eine Reihe von Vorteilen:

\begin{description}
\item[Flexible Konfiguration] \hfill \\
Der Anwender erhält erhöhte Kontrolle über das Conversation Routing in seinem Contact Center. Die Features, die in der neuen DSL angeboten werden (das Abspielen von Audiodateien etc.) sind zwar nicht neu. Aber nun kann flexibler auf diese zugegriffen werden und mehr Anforderungen von Contact Centern können erfüllt werden.
\item[Unabhängigkeit gegenüber Drittanbietern] \hfill \\
Vor der Implementierung der DSL lief das Conversation Routing über Software von Drittanbietern ab. Zum Beispiel werden eingehende Anrufe über eine Scripting-Schnittstelle der Telefonanlage gesteuert. Mit einer eigenständigen DSL erhält MyCC eine neue Unabhängigkeit gegenüber den Beschränkungen und Nachteilen dieser Drittanbieter-Software.
\item[Hohe Skalierbarkeit] \hfill \\
Die Ausführung von DSL-Skripten wird von einem dediziertem Dienst, der Conversation Routing Engine, übernommen (Detail folgen in Kapitel [LINK]). Dies entlastet den MyCC-Server und erlaubt das parallele Aufschalten von mehreren Conversation Routing Engines bei hoher Systemlast. So kann die generelle Performanz von MyCC verbessert werden.
\end{description}

\section{Domänenspezifische Sprachen}
TODO

\section{Benötigtes Vorwissen}
TODO

\subsection{.NET}
TODO 
 
\subsection{Roslyn}
TODO

\subsection{Asynchrone Methodenausführung in .NET}
TODO

\subsection{SIP}
TODO