\chapter{Zusammenfassung und Ausblick}
Für den Produktivbetrieb als Teil des Produktes MyContactCenter braucht die vorliegend implementierte domänenspezifische Sprache noch Anpassungen, vor allem im Bereich der umliegenden Infrastruktur. 
Dennoch ist die DSL bereits weit fortgeschritten. Der Entwurf der Sprache geht mit der Abdeckung der eingangs erwähnten Anforderungen einen großen Schritt auf die Produkt-Vision einer freien Konversationsroutingprogrammierung durch den Benutzer zu. Diesem Entwurf folgt die Implementierung von drei Komponenten, die für eine Umsetzung der DSL nötig sind. Eine dieser Komponenten ist der Editor, der dank flexibler MVVM-Struktur ein funktionsfähiger Prototyp der finalen Benutzeroberfläche ist. Eine weitere ist der Transformator, der sich die Roslyn-API zu Nutze macht, um lauffähige MSIL-Syntax zu generieren und zu kompilieren, um anschließend eine betriebsfähige Assembly an die Routing Engine auszuliefern. Die dritte Komponente ist die  Modell-Validierung, welche dem Benutzer während der Modellierung laufend Feedback gibt und dabei hilft, etwaige Fehlerzustände frühzeitig zu beheben. 
\newline
Zusammen machen es diese drei Komponenten bereits möglich, erste funktionstüchtige Konversationsroutings zu spezifizieren. Wie manuelle und automatisierte Tests zeigen, können diese auch erfolgreich ausgeführt werden. Neben den funktionalen Anforderungen erweist sich das System außerdem als performant. Die Implementierung stützt sich dabei auf gradlinigen Code, welcher die Wartbarkeit des Systems auch für zukünftige Erweiterungen sicher stellt. 
\newline
Für solche bietet das System auch in Zukunft Raum: So ist unter anderem die Unterstützung von anderen Kontaktmöglichkeiten wie Emails oder Web Chats für Konversationsroutings in Planung. Zusätzlich kann der eigentliche Umgang mit einem Kontakt erweitert werden, zum Beispiel indem dem Benutzer die Möglichkeit gegeben wird, nach der Zustellung zu einem Agenten weiterhin mit dem Anruf zu interagieren. Denkbar ist dies durch die Einführung von Event-Instruktionen, welche ausgeführt werden, wenn der Agent den Ruf annimmt oder anderweitige Ereignisse auftreten. Nicht zuletzt das Repertoire an Instruktionen bietet weiteren Implementierungsspielraum: Instruktionen welche das Anreichern von Anrufen mit Anrufer-spezifischen Daten erlauben böten dem Benutzer einen erheblichen Mehrwert. Alles in Allem bietet die umgesetzte domänenspezifische Sprache eine erste Implementierung mit Erweiterungspotential und damit eine solide Basis zur weiteren 
Umsetzung eines integralen Features von MyContactCenter.