\kurzfassung

%% deutsch
\paragraph*{}
In der vorliegenden Ausarbeitung wird der Entwurf und die anschließende Implementierung einer grafischen domänenspezifischen Sprache dokumentiert, mit der das Verhalten eines Contactcenters für eingehende Konversationen modelliert werden kann. Dabei wird nach einer Einleitung und der Erläuterung der Motivation auf die Sprache und ihre Elemente eingegangen. Anschließend wird die Umsetzung ihrer Software-Komponenten besprochen. Diese bestehen aus einem grafischen Editor zum Erstellen von Modellen in der Sprache und einem Transformator, welcher ein so erstelltes Modell in ausführbaren CIL-Bytecode konvertiert. Die Arbeit schließt mit einer Beschreibung, wie das implementierte System getestet wurde, einer Erörterung der Performanz und Komplexität des geschriebenen Codes sowie einem Fazit, in dem ein Ausblick auf kommende Funktionen der Sprache gegeben wird.